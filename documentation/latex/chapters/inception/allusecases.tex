\قسمت{لیست اولویت‌بندی شده نیازمندی‌های وظیفه‌ای}

در این قسمت لیست اولویت‌بندی شده‌ی نیازمندی‌ها با روش \لر{MoSCoW} آورده شده است. در واقع \لر{MoSCoW} میزان اهمیت هر کدام از نیازمندی‌ها را با روش زیر مشخص می‌کند:

۱) \مهم{\لر{M}}\زیرنویس{\لر{Must be implemented}}: نیازمندی‌هایی که حتما باید پیاده‌سازی شوند و در صورت پیاده‌سازی نشدن پروژه با شکست مواجه شده است.

۲) \مهم{\لر{S}}\زیرنویس{\لر{Should be implemented}}: نیازمندی‌هایی که باید پیاده‌سازی شوند و نیازمندی‌های مهمی هستند اما در صورت پیاده‌سازی نشدن اختلالی در پروژه به وجود نمی‌آید.

۳) \مهم{\لر{C}}\زیرنویس{\لر{Could be implemented}}: نیازمندی‌هایی که در صورت وجود وقت و زمان کافی بهتر است پیاده‌سازی شوند.

۴) \مهم{\لر{W}}\زیرنویس{\لر{Won't be implemented}}: نیازمندی‌هایی که نباید پیاده‌سازی شوند.

\شروع{فقرات}
	\فقره \مهم{بخش امور کاربری}
	\شروع{فقرات}
		\فقره افزودن مدیر احراز هویت (\لر{M})
		\فقره افزودن مدیر سامانه (\لر{S})
		\فقره ثبت‌نام کردن صاحب بار (\لر{M})
		\فقره ثبت‌نام کردن راننده (\لر{M})
		\فقره وارد شدن مدیر سامانه (\لر{M})
		\فقره وارد شدن راننده در برنامه موبایل (\لر{M})
		\فقره وارد شدن مدیر احراز هویت (\لر{M})
		\فقره وارد شدن راننده در سایت (\لر{M})
		\فقره وارد شدن صاحب بار (\لر{M})
		\فقره خروج از سامانه (\لر{S})
	\پایان{فقرات}
	\فقره \مهم{بخش راننده}
	\شروع{فقرات}
		\فقره مشاهده تاریخچه بار (\لر{C})
		\فقره مشاهده اطلاعات بار (\لر{M})
		\فقره چاپ کردن بارنامه (\لر{M})
		\فقره اعلام کردن حادثه (\لر{M})
	\پایان{فقرات}
	
	\فقره \مهم{بخش زمان}
	\شروع{فقرات}
		\فقره به‌روزرسانی موقعیت جغرافیایی بار (\لر{M})
	\پایان{فقرات}
	\فقره \مهم{بخش مدیریت}
	\شروع{فقرات}
		\فقره تایید کردن اطلاعات راننده (\لر{M})
		\فقره تخصیص خودرو حمل بار به بار (\لر{M})
		\فقره پرداخت حقوق رانندگان (\لر{C})
		\فقره اختصاص بار به راننده (\لر{M})
		\فقره ویرایش اطلاعات راننده (\لر{S})
		\فقره تعیین وضعیت راننده		 (\لر{M})
		\فقره مشاهده موقعیت جغرافیایی بار (\لر{M})
		\فقره ثبت خودرو حمل بار جدید (\لر{M})
		\فقره مشاهده وضعیت خودرو حمل بار (\لر{M})
		\فقره مشاهده کردن اطلاعات بار (\لر{M})
		\فقره مشاهده کردن اطلاعات خودرو حمل بار (\لر{M})
		\فقره مشاهده کردن اطلاعات راننده (\لر{M})
		\فقره ثبت کردن سفارش و اختصاص کد (\لر{M})
		\فقره حذف کردن سفارش (\لر{M})
		\فقره ویرایش کردن سفارش (\لر{M})
		\فقره مشاهده اطلاعات سفارش (\لر{M})
		\فقره مشاهده رتبه‌بندی رانندگان (\لر{M})
		
	\پایان{فقرات}
	\فقره \مهم{بخش سفارش}
	\شروع{فقرات}
		\فقره مشاهده تاریخچه سفارش‌ها (\لر{C})
		\فقره مشاهده اطلاعات سفارش (\لر{M})
		\فقره تایید تحویل بار (\لر{M})
		\فقره مشاهده موقعیت جغرافیایی بار (\لر{M})
		\فقره مشاهده کردن اطلاعات بار (\لر{M})
		\فقره ثبت امتیاز (\لر{M})
	\پایان{فقرات}
\پایان{فقرات}

\newpage
\قسمت{لیست نیازمندی‌های غیروظیفه‌ای}

\شروع{فقرات}
	\فقره \مهم{واسط کاربری}
	
	\شروع{فقرات}
		\فقره به زبان فارسی باشد.
		\فقره قسمت طراحی شده برای مدیر سامانه و صاحب بار و کاربران دیگر سایت، باید منطبق با واسط‌های کاربری متداول، طراحی شده و کار کردن با آن راحت باشد.
		\فقره طراحی واسط کاربری باید در کل سامانه تحت وب و هم‌چنین در برنامه موبایل یک‌نواخت باشد.
		\فقره برای انجام هیچ مورد کاربرد نباید کاربر ناچار شود بیش از ۵ عدد کلیک انجام دهد.
		\فقره باید برنامه موبایل و سامانه تحت وب هر دو طوری ساختاردهی شده باشند که کاربرپسند\زیرنویس{\لر{User-friendly}} باشد.
	\پایان{فقرات}
	\فقره \مهم{آموزش و راهنمایی}
	\شروع{فقرات}
		\فقره آموزش‌هایی برای مدیران سامانه باید در مورد چگونگی استفاده از سیستم در اختیار آن‌ها قرار گیرد.
		\فقره برای همه‌ی کاربران‌ (صاحب بار، راننده، مدیر سامانه و مدیر احراز هویت) باید راهنمای کاربری ساخته شود.
		\فقره سامانه باید دارای یک راهنمای عملیاتی برای نصب باشد.
	\پایان{فقرات}
	\فقره \مهم{کارایی}
	\شروع{فقرات}
		\فقره سامانه تحت وب و برنامه موبایل باید به سرعت به درخواست و نیازهای کاربران پاسخ دهد. این محدودیت در حدود نهایتا ۵ ثانیه است.
	\پایان{فقرات}
	\فقره \مهم{اطمینان}
	\شروع{فقرات}
		\فقره سیستم در طول ۲۴ ساعت شبانه‌روز حداکثر ۳ دقیقه می‌تواند فعال نباشد.
		\فقره نرخ بروز خطا باید از ۱ درصد کم‌تر باشد.
		\فقره هر دو خطای متوالی باید بیش از ۵ ساعت فاصله داشته باشند.
		\فقره سامانه باید یک سرور پشتیبان داشته باشد که همواره به‌روزرسانی شود تا در صورت خطا از مانایی و سازگاری داده‌ها محافظت شود.
		\فقره اتفاقاتی که بعد از ارتباط با سامانه برای کاربران سایت اتفاق می‌افتند باید قابل پیش‌بینی باشند.
	\پایان{فقرات}
	\فقره \مهم{امنیت}
	\شروع{فقرات}
		\فقره اطلاعات تمام کنش‌گرهای سامانه باید فقط در اختیار خودشان باشد و هیچ‌کدام حق دسترسی بیش از میزان تعریف‌شده نداشته باشند.
	\پایان{فقرات}
	\فقره \مهم{سیستم‌عامل}
	\شروع{فقرات}
		\فقره سرور مورد استفاده ما باید قابلیت پشتیبانی سیستم‌عامل اوبونتو را داشته باشد.
	\پایان{فقرات}
	\فقره \مهم{نگهداری}
	\شروع{فقرات}
		\فقره نرم‌افزار باید با اصول شی‌گرا ساخته شده و مولفه‌های جداگانه داشته باشد.
		\فقره نام‌گذاری‌ها در کد منبع باید خوانا باشد و با اصول شی‌گرا و زبان مبدا هم‌خوانی داشته باشد. همین‌طور بهتر است کد منبع پروژه، مستند و خوانا باشد.
		\فقره از آن‌جایی که اکثر کدهای استفاده شده به زبان پایتون است، بهتر است تمامی اعضای گروه با اصول استاندارد کدزنی به زبان پایتون آشنا باشند.
	\پایان{فقرات}
	\فقره \مهم{توسعه‌پذیری}
	\شروع{فقرات}
		\فقره بهتر است مسئله برای حالت کلی حل شود.
	\پایان{فقرات}
	\فقره \مهم{انتقال‌پذیری}
	\شروع{فقرات}
		\فقره همه‌ی فایل‌های اطلاعاتی ذخیره‌شده از سامانه (نظیر اطلاعات بارها، رانندگان، مدیران سامانه و ...) باید قابل انتقال به رایانه یا سرور دیگر باشند.
	\پایان{فقرات}
	\فقره \مهم{تطبیق‌پذیری}
	\شروع{فقرات}
		\فقره پذیرش هرگونه تغییر جدید که در سامانه ممکن است به وجود بیاید.
	\پایان{فقرات}
	\فقره \مهم{حق استفاده و انتشار سامانه}
	\شروع{فقرات}
		\فقره حق استفاده از نسخه‌های مختلف سامانه تنها برای صاحبان قانونی نرم‌افزار امکان‌پذیر است.
	\پایان{فقرات}
\پایان{فقرات}