\قسمت{لیست اولویت‌بندی شده ریسک‌ها}

 در این قسمت مواردی که ممکن است برای انجام پروژه مشکل‌زا باشند، آورده شده است.

\مهم{شرح ریسک:} کافی نبودن منابع انسانی برای محقق کردن تمامی نیازمندی‌ها

\مهم{محدوده ریسک:} فنی در مراحل طراحی و پیاده‌سازی


\مهم{میزان اهمیت ریسک:} بالا

\مهم{احتمال وقوع ریسک:} بالا. با توجه به گسترده و بزرگ بودن پروژه و پیچیدگی‌های فنی آن، احتمال مواجه شدن با خطاهای متعدد در آن بسیار زیاد است و این امر منجر به صرف زمان، هزینه و انرژی می‌شود.

\مهم{راه‌حل پیش‌گیرانه:} برای پیش‌گیری از این مشکل، می‌توان ابتدا نیازمندی‌های از نوع اجباری در مدل \لر{MoSCoW} را پیاده‌سازی کرد و سپس به نیازمندی‌های دیگر پرداخت.

\newpage

\مهم{شرح ریسک:} آشنا نبودن برخی از اعضای تیم با تکنولوژی‌های ضروری و در نتیجه زمان‌بر شدن پروسه یادگیری که باعث می‌شود نیازمندی‌ها در زمان‌های پیش‌بینی شده انجام نشوند.

\مهم{محدوده ریسک:} فنی در مراحل طراحی و پیاده‌سازی

\مهم{میزان اهمیت ریسک:} بالا / متوسط

\مهم{احتمال وقوع ریسک:} این ریسک به احتمال قوی اتفاق خواهد افتاد اما تعداد دفعات آن بستگی به گستردگی دانش اعضای تیم دارد.

\مهم{راه‌حل پیش‌گیرانه:} برای پیش‌گیری از مشکلات ناشی از این ریسک می‌توان ددلاین‌ها را کمی جلوتر از زمان واقعی آن‌ها تعریف کرد تا در صورت نیاز به یادگیری، زمان کافی تا ددلاین‌های اصلی وجود داشته باشد.

\newpage

\مهم{شرح ریسک:} سخت بودن کار گروهی و ارتباط بین اعضای تیم در شرایطی که امکان ملاقات حضوری به هیچ‌وجه وجود ندارد. این ریسک باعث می‌شود تا توافق اعضای تیم بر سر موضوعات مهم به‌کندی صورت پذیرد.

\مهم{محدوده ریسک:} فنی و بیش‌تر در مرحله طراحی

\مهم{میزان اهمیت ریسک:} بالا

\مهم{احتمال وقوع ریسک:} این ریسک در حال حاضر وجود دارد و امکان بازگشت به شرایط عادی نیز بعید به نظر می‌رسد.

\مهم{راه‌حل پیش‌گیرانه:} برای حل نسبی مشکل ارتباط، می‌توان از ابزارها و نرم‌افزارهای ارتباطی موجود از جمله تلگرام و یا اسکایپ استفاده کرد.

\newpage


\مهم{شرح ریسک:} احتمال به وجود آمدن سوء تفاهم در مورد جزئیات پروژه با توجه به این‌که صورت پروژه تعریف‌شده خیلی کوتاه بوده و تنها در یک صفحه آورده شده، بالا است. امکان در نظر گرفته نشدن برخی امکانات و هم‌چنین مبهم بودن امکانات نوشته شده وجود دارد. این ریسک در صورت وقوع می‌تواند باعث تحلیل اشتباه اعضای فنی تیم شده و در فازهای بعدی پروژه وقت زیادی صرف اصلاح و یا حتی پیاده‌سازی دوباره شود.

\مهم{محدوده ریسک:} فنی و در مرحله طراحی و پیاده‌سازی

\مهم{میزان اهمیت ریسک:} بالا

\مهم{احتمال وقوع ریسک:} بالا

\مهم{راه‌حل پیش‌گیرانه:}  برای پیش‌گیری از وقوع این ریسک باید ارتباط با مشتری بیش‌تر شده و با گفت‌وگوی اعضای تیم با مشتریان، ابهامات موجود قبل از پیاده‌سازی و در مرحله طراحی برطرف شود.
