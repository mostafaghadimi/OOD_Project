\قسمت{مشخصات موارد کاربرد}

 در این قسمت لیست توضیحات، نحوه شروع و  روندهای جایگزین موارد کاربرد آورده شده است.

\مهم{مورد کاربرد:}
افزودن مدیر احراز هویت

\مهم{شماره:}
۱

\مهم{عامل اصلی:}
مدیر سامانه

\مهم{عامل فرعی:}
ندارد

\مهم{شرایط اولیه:}
مدیر سامانه، وارد سامانه شده باشد.

\مهم{روند اصلی:}

۱. این مورد کاربرد وقتی آغاز می‌شود که مدیر سامانه بخواهد یک مدیر احراز هویت جدید به سامانه اضافه کند.

۲. مدیر سامانه اطلاعات مربوط به مدیر احراز هویت جدید را وارد می‌کند. این اطلاعات شامل نام، نام‌خانوادگی، شناسه‌کاربری و  رمز عبور می‌باشد.

۳. مدیر سامانه اطلاعات مدیر احراز هویت جدید را در سامانه ثبت می‌کند.

\مهم{شرایط نهایی:}
افزوده شدن مدیر احراز هویت جدید

\مهم{روند جایگزین:}
ندارد

\newpage

\مهم{مورد کاربرد:}
افزودن مدیر سامانه

\مهم{شماره:}
۲

\مهم{عامل اصلی:}
مدیر سامانه

\مهم{عامل فرعی:}
ندارد

\مهم{شرایط اولیه:}
مدیر سامانه، وارد سامانه شده باشد.

\مهم{روند اصلی:}

۱. این مورد کاربرد وقتی آغاز می‌شود که مدیر سامانه بخواهد یک مدیر سامانه‌ی جدید ایجاد کند.

۲. مدیر سامانه اطلاعات مدیر سامانه‌ی جدید را وارد می‌کند. این اطلاعات شامل نام، نام‌خانوادگی، شناسه کاربری، رمز عبور و شماره تماس می‌باشد.

۳. مدیر سامانه اطلاعات مدیر سامانه جدید را ثبت می‌کند.

\مهم{شرایط نهایی:}
افزوده شدن مدیر سامانه جدید

\مهم{روند جایگزین:}
ندارد


\newpage

\مهم{مورد کاربرد:}
ثبت‌نام کردن صاحب بار

\مهم{شماره:}
۳

\مهم{عامل اصلی:}
صاحب بار

\مهم{عامل فرعی:}
ندارد

\مهم{شرایط اولیه:}
ندارد

\مهم{روند اصلی:}

۱. این مورد کاربرد زمانی شروع می‌شود که بازدیدکننده سایت بخواهد در سامانه به عنوان صاحب بار ثبت‌نام کند.

۲. صاحب بار اطلاعات مربوط به خودش را در سامانه وارد می‌کند. این اطلاعات شامل نام، نام‌خانوادگی، شناسه کاربری، رمز عبور و شماره تماس می‌باشد.

۳. صاحب بار اطلاعاتش را در سامانه ثبت می‌کند.

\مهم{شرایط نهایی:}
افزوده شدن یک صاحب بار جدید

\مهم{روند جایگزین:}
ندارد

\newpage

\مهم{مورد کاربرد:}
ثبت‌نام کردن راننده

\مهم{شماره:}
۴

\مهم{عامل اصلی:}
راننده

\مهم{عامل فرعی:}
ندارد

\مهم{شرایط اولیه:}
ندارد

\مهم{روند اصلی:}

۱. این مورد کاربرد هنگامی شروع می‌شود که بازدید کننده سایت بخواهد به عنوان راننده در سامانه ثبت‌نام کند.

۲. راننده اطلاعات خود را در سامانه وارد می‌کند. این اطلاعات شامل نام، نام خانوادگی، شماره تماس، شماره شناسنامه و عکس پرسنلی است.

۳. راننده اطلاعات خود را در سامانه ثبت می‌کند.

\مهم{شرایط نهایی:}
ثبت اطلاعات متقاضی برای نقش راننده

\مهم{روند جایگزین:}
ندارد

\newpage

\مهم{مورد کاربرد:}
وارد شدن مدیر سامانه

\مهم{شماره:}
۵

\مهم{عامل اصلی:}
مدیر سامانه

\مهم{عامل فرعی:}
ندارد

\مهم{شرایط اولیه:}
ندارد

\مهم{روند اصلی:}

۱. این مورد کاربرد زمانی شروع می‌شود که مدیر سامانه بخواهد وارد سامانه شود.

۲. مدیر سامانه اطلاعات کاربری خود را وارد می‌کند. اطلاعات کاربری شامل شناسه کاربری و رمز عبور می‌باشد.

۳. مدیر سامانه درخواست ورود به سایت را می‌دهد.

\مهم{شرایط نهایی:}
ورود مدیر سامانه به بخش مدیریت

\مهم{روند جایگزین:}
عدم صحت اطلاعات

\newpage

\مهم{مورد کاربرد:}
روند جایگزین مورد کاربرد: وارد شدن مدیر سامانه: عدم صحت اطلاعات

\مهم{شماره:}
۵.۱

\مهم{عامل اصلی:}
مدیر سامانه

\مهم{عامل فرعی:}
ندارد

\مهم{شرایط اولیه:}
شناسه کاربری یا رمز عبور اشتباه وارد شده باشد.

\مهم{روند اصلی:}

۱. این مورد کاربرد هنگامی شروع می‌شود که شناسه کاربری یا رمز عبور اشتباه وارد شده باشد.

۲. پیغامی مبنی بر عدم صحت اطلاعات به مدیر سامانه نمایش داده می‌شود.

\مهم{شرایط نهایی:}
ندارد

\مهم{روند جایگزین:}
ندارد

\مهم{نکته:} به دلیل شباهت بسیار زیاد و هم‌چنین واضح بودن دیگر موارد کاربرد وارد شدن به سامانه و هم‌چنین روند جایگزین آن‌ها از آوردن آن‌ها در این قسمت اجتناب می‌کنیم.

\newpage

\مهم{مورد کاربرد:}
خروج از سامانه

\مهم{شماره:}
۶

\مهم{عامل اصلی:}
کاربر سایت

\مهم{عامل فرعی:}
ندارد

\مهم{شرایط اولیه:}
کاربر سایت وارد سامانه شده باشد

\مهم{روند اصلی:}

۱. این مورد کاربرد هنگامی فعال می‌شود که کاربر سایت بخواهد از سیستم خارج شود.

۲. کاربر سایت درخواست خروج را ارسال می‌کند.

\مهم{شرایط نهایی:}
خروج کاربر سایت از سامانه 

\مهم{روند جایگزین:}
ندارد

\newpage

\مهم{مورد کاربرد:}
چاپ کردن اطلاعات بارنامه

\مهم{شماره:}
۷

\مهم{عامل اصلی:}
راننده

\مهم{عامل فرعی:}
ندارد

\مهم{شرایط اولیه:}
راننده وارد سامانه شده باشد.

\مهم{روند اصلی:}

۱. این مورد کاربرد هنگامی شروع می‌شود که راننده بخواهد اطلاعات مربوط به یک بارنامه را چاپ کند. 

۲. راننده یک بار (بار موجود) را انتخاب کرده و درخواست مشاهده‌ی اطلاعات آن را ارسال می‌کند.

۳. اطلاعات مربوط به بار به راننده نمایش داده می‌شود.

۴. راننده این اطلاعات را می‌تواند به چاپ کند.

\مهم{شرایط نهایی:}
چاپ شدن اطلاعات بارنامه

\مهم{روند جایگزین:}
ندارد

\newpage

\مهم{مورد کاربرد:}
اعلام کردن حادثه

\مهم{شماره:}
۸

\مهم{عامل اصلی:}
راننده

\مهم{عامل فرعی:}
ندارد

\مهم{شرایط اولیه:}

راننده وارد سامانه شده باشد.

حادثه‌ای در خلال حمل‌و‌نقل بار برای راننده و یا خودروی حمل‌ بار اتفاق افتاده باشد.

\مهم{روند اصلی:}

۱. این مورد کاربرد هنگامی فعال می‌شود که برای راننده و یا خودروی حمل بار اتفاقی رخ داده باشد و  راننده بخواهد آن را گزارش کند.

۲. راننده اعلام حادثه را انتخاب می‌کند.

۳. شرح حادثه و محل وقوع آن را وارد می‌کند.

۴. با ثبت کردن موارد مذکور، حادثه را در سامانه ثبت می‌کند.

\مهم{شرایط نهایی:}
اعلام کردن حادثه پیش‌آمده

\مهم{روند جایگزین:}
ندارد

\newpage

\مهم{مورد کاربرد:}
به‌روزرسانی موقعیت جغرافیایی بار

\مهم{شماره:}
۹

\مهم{عامل اصلی:}
زمان

\مهم{عامل فرعی:}
ندارد

\مهم{شرایط اولیه:}
ندارد

\مهم{روند اصلی:}

۱. این مورد کاربرد در بازه‌های زمانی مشخصی شروع به کار می‌کند.

۲. موقعیت جغرافیایی بار نظیر طول و عرض جغرافیایی به‌روز می‌شوند.

\مهم{شرایط نهایی:}
به‌روز شدن مؤلفه‌های موقعیت جغرافیایی	

\مهم{روند جایگزین:}
ندارد

\newpage

\مهم{مورد کاربرد:}
تایید کردن اطلاعات راننده

\مهم{شماره:}
۱۰

\مهم{عامل اصلی:}
مدیر احراز هویت

\مهم{عامل فرعی:}
ندارد

\مهم{شرایط اولیه:}
مدیر احراز هویت وارد سامانه شده باشد.

\مهم{روند اصلی:}

۱. این مورد کاربرد هنگامی شروع به کار می‌کند که مدیر احراز هویت یکی از متقاضیان رانندگی در سامانه را انتخاب می‌کند.

۲. مدیر احراز هویت اطلاعات وارد شده توسط متقاضی رانندگی در سامانه را به دور دقیق چک می‌کند.

۲.۱. اگر اطلاعات وارد شده مطابق با اطلاعات حقیقی راننده باشد، هویت او را تایید کرده و به عنوان راننده تایید می‌شود و یک شناسه کاربری و رمز عبور به او اختصاص داده می‌شود.

۲.۲ اگر اطلاعات وارد شده درست نباشد، آن‌گاه درخواست او برای رانندگی در سامانه حذف می‌شود.

\مهم{شرایط نهایی:}
تایید کردن صحت اطلاعات راننده

\مهم{روند جایگزین:}
ندارد

\newpage

\مهم{مورد کاربرد:}
تخصیص خودرو حمل بار به بار

\مهم{شماره:}
۱۱

\مهم{عامل اصلی:}
مدیر سامانه

\مهم{عامل فرعی:}
ندارد

\مهم{شرایط اولیه:}

مدیر سامانه وارد سامانه شده باشد. 

خودروی حمل‌باری در گاراژ موجود باشد.

سفارشی در سیستم موجود باشد.

\مهم{روند اصلی:}

۱. این مورد کاربرد هنگامی شروع می‌شود که مدیرسامانه بخواهد به یکی از سفارش‌ها ماشینی اختصاص دهد.

۲. مدیر سامانه ابتدا یکی از سفارش‌ها را انتخاب می‌کند.

۳. سپس یک خودروی حمل بار موجود در گاراژ را به آن اختصاص می‌دهد.

۴. سفارش را تبدیل به بار می‌کند.

\مهم{شرایط نهایی:}
اختصاص یافتن بار به خودروی حمل بار

\مهم{روند جایگزین:}
ندارد

\newpage

\مهم{مورد کاربرد:}
ویرایش اطلاعات راننده

\مهم{شماره:}
۱۲

\مهم{عامل اصلی:}
مدیر سامانه

\مهم{عامل فرعی:}
ندارد

\مهم{شرایط اولیه:}
مدیر سامانه وارد سامانه شده باشد.

\مهم{روند اصلی:}

۱. شامل(مشاهده کردن اطلاعات راننده)

۲. مدیر سامانه ویرایش را برمی‌گزیند.

۳. اطلاعات که می‌خواهد ویرایش کند را وارد می‌کند.

۴. تغییرات را ذخیره می‌کند.

\مهم{شرایط نهایی:}
اطلاعات راننده ویرایش شود

\مهم{روند جایگزین:}
ندارد

\newpage

\مهم{مورد کاربرد:}
اختصاص دادن بار به راننده

\مهم{شماره:}
۱۳

\مهم{عامل اصلی:}
مدیر سامانه

\مهم{عامل فرعی:}
ندارد

\مهم{شرایط اولیه:}

مدیر سامانه وارد سامانه شده باشد.

باری موجود باشد.

راننده‌ای در حالت آزاد وجود داشته باشد.

\مهم{روند اصلی:}

۱. این مورد کاربرد هنگامی شروع می‌شود که مدیر سامانه بخواهد به راننده‌ای، بار اختصاص دهد.

۲. مدیر سامانه یکی از بارها را انتخاب می‌کند.

۳. برای بار انتخاب شده، یکی از راننده‌ها را انتخاب می‌کند.

\مهم{شرایط نهایی:}
بار به راننده اختصاص یابد

\مهم{روند جایگزین:}
ندارد

\newpage

\مهم{مورد کاربرد:}
تعیین وضعیت راننده

\مهم{شماره:}
۱۴

\مهم{عامل اصلی:}
مدیر سامانه

\مهم{عامل فرعی:}
ندارد

\مهم{شرایط اولیه:}
مدیر سامانه وارد سامانه شده باشد.

\مهم{روند اصلی:}

۱. شامل (مشاهده کردن اطلاعات راننده)

۱.۱ اگر حادثه‌ای توسط راننده گزارش شده باشد یا مدیر سامانه به راننده بار تخصیص بدهد و یا صاحب بار وضعیت بار را مشخص کند، مدیر سامانه وضعیت راننده را تغییر می‌دهد.

\مهم{شرایط نهایی:}
وضعیت راننده تغییر یابد.

\مهم{روند جایگزین:}
ندارد


\newpage

\مهم{مورد کاربرد:}
مشاهده موقعیت جغرافیایی بار

\مهم{شماره:}
۱۵

\مهم{عامل اصلی:}
مدیر سامانه

\مهم{عامل فرعی:}
ندارد

\مهم{شرایط اولیه:}
مدیر سامانه وارد سامانه شده باشد.

\مهم{روند اصلی:}

۱. شامل(مشاهده کردن اطلاعات بار)

۲. مدیر سامانه مشاهده‌ی موقعیت بار را برمی‌گزیند.

۳. طول و عرض جغرافیایی و موقعیت مکانی بار نمایش داده می‌شود.

\مهم{شرایط نهایی:}
موقعیت جغرافیایی بار روی نقشه مشخص باشد.

\مهم{روند جایگزین:}
ندارد

\newpage

\مهم{مورد کاربرد:}
ثبت خودرو حمل بار جدید

\مهم{شماره:}
۱۶

\مهم{عامل اصلی:}
مدیر سامانه

\مهم{عامل فرعی:}
ندارد

\مهم{شرایط اولیه:}
مدیر سامانه وارد سامانه شده باشد.

\مهم{روند اصلی:}

۱. این مورد کاربرد هنگامی شروع می‌شود که مدیر سامانه بخواهد خودرو حمل بار جدیدی را ثبت کند.

۲. به همین منظور ثبت خودروی حمل بار جدید را برمی‌گزیند.

۳. اطلاعات مربوط به خودرو حمل بار را وارد می‌کند.

۴. اطلاعات خودرو حمل بار را ثبت می‌کند.

\مهم{شرایط نهایی:}
خودرو حمل بار جدید در سامانه ثبت شود.

\مهم{روند جایگزین:}
ندارد

\newpage

\مهم{مورد کاربرد:}
مشاهده وضعیت خودرو حمل بار

\مهم{شماره:}
۱۷

\مهم{عامل اصلی:}
مدیر سامانه

\مهم{عامل فرعی:}
ندارد

\مهم{شرایط اولیه:}
مدیر سامانه وارد سامانه شده باشد.

\مهم{روند اصلی:}

۱. شامل (مشاهده کردن اطلاعات خودرو حمل بار)

۲. مدیر سامانه مشاهده وضعیت خودرو حمل بار را انتخاب می‌کند.

۳. وضعیت خودروی حمل بار نمایش داده می‌شود.

\مهم{شرایط نهایی:}
وضعیت خودرو حمل بار نمایش داده شود.

\مهم{روند جایگزین:}
ندارد

\newpage

\مهم{مورد کاربرد:}
مشاهده کردن اطلاعات بار

\مهم{شماره:}
۱۸

\مهم{عامل اصلی:}
مدیر سامانه

\مهم{عامل فرعی:}
ندارد

\مهم{شرایط اولیه:}
مدیر سامانه وارد سامانه شده باشد.

\مهم{روند اصلی:}

۱. این مورد کاربرد هنگامی شروع می‌شود که مدیر سامانه بخواهد اطلاعات مربوط بار را مشاهده کند.

۲. مدیر سامانه یکی از بارها را انتخاب می‌کند.

۳. اطلاعات مربوط به بار نمایش داده می‌شود.

\مهم{شرایط نهایی:}
اطلاعات مربوط به بار نمایش داده شود.

\مهم{روند جایگزین:}
ندارد

\newpage

\مهم{مورد کاربرد:}
مشاهده کردن اطلاعات خودرو حمل بار

\مهم{شماره:}
۱۹

\مهم{عامل اصلی:}
مدیر سامانه

\مهم{عامل فرعی:}
ندارد

\مهم{شرایط اولیه:}
مدیر سامانه وارد سامانه شده باشد.

\مهم{روند اصلی:}

۱. این مورد کاربرد هنگامی شروع می‌شود که مدیر سامانه بخواهد اطلاعات مربوط به خودرو حمل بار را مشاهده کند.

۲. یکی از خودروهای حمل بار را انتخاب می‌کند.

۳. اطلاعات خودروی حمل بار به او نمایش داده می‌شود.

\مهم{شرایط نهایی:}
نمایش اطلاعات خودرو حمل بار

\مهم{روند جایگزین:}
ندارد

\newpage



\مهم{مورد کاربرد:}
مشاهده کردن اطلاعات راننده

\مهم{شماره:}
۲۰

\مهم{عامل اصلی:}
مدیر سامانه

\مهم{عامل فرعی:}
ندارد

\مهم{شرایط اولیه:}
مدیر سامانه وارد سامانه شده باشد

\مهم{روند اصلی:}

۱. این مورد کاربرد هنگامی شروع می‌شود که مدیر سامانه بخواهد اطلاعات یک راننده را مشاهده کند.

۲. مدیر سامانه یکی از راننده‌ها را انتخاب می‌کند.

۳. اطلاعات مربوط به راننده برای او نمایش داده می‌شود.

\مهم{شرایط نهایی:}
مشاهده تمامی اطلاعات راننده

\مهم{روند جایگزین:}
ندارد

\newpage

\مهم{مورد کاربرد:}
ثبت کردن سفارش جدید

\مهم{شماره:}
۲۱

\مهم{عامل اصلی:}
مدیر سامانه

\مهم{عامل فرعی:}
ندارد

\مهم{شرایط اولیه:}
مدیر سامانه وارد سامانه شده باشد.

\مهم{روند اصلی:}

۱. این مورد کاربرد زمانی شروع می‌شود که مدیر سامانه بخواهد سفارش جدیدی را ثبت کند.

۲. مدیر سامانه ثبت سفارش جدید را انتخاب می‌کند.

۳. اطلاعات مربوط به سفارش را نظیر مبدا، مقصد، وزن و … را کامل می‌کند.

۴. اطلاعات مربوط به سفارش را ذخیره می‌کند.

\مهم{شرایط نهایی:}
سفارش جدید در سامانه ثبت شود.

\مهم{روند جایگزین:}
ندارد 


\newpage

\مهم{مورد کاربرد:}
ویرایش کردن سفارش

\مهم{شماره:}
۲۲

\مهم{عامل اصلی:}
مدیر سامانه

\مهم{عامل فرعی:}
ندارد

\مهم{شرایط اولیه:}

مدیر سامانه وارد سامانه شده باشد.

سفارشی در سامانه وجود داشته باشد.

\مهم{روند اصلی:}

۱. شامل (مشاهده اطلاعات سفارش)

۲. اطلاعاتی که باید ویرایش شوند، اضافه می‌شوند.

۳. اطلاعات در سامانه ثبت می‌شود.

\مهم{شرایط نهایی:}
اطلاعات سفارش ویرایش شود.

\مهم{روند جایگزین:}
ندارد

\newpage

\مهم{مورد کاربرد:}
حذف کردن سفارش

\مهم{شماره:}
۲۳

\مهم{عامل اصلی:}
مدیر سامانه

\مهم{عامل فرعی:}
ندارد

\مهم{شرایط اولیه:}
مدیر سامانه وارد سامانه شود.

سفارشی در سامانه وجود داشته باشد.

\مهم{روند اصلی:}

۱. شامل (مشاهده اطلاعات سفارش)

۲. مدیر سامانه حذف را برمی‌گزیند.

۳. سفارش از سامانه حذف می‌شود.

\مهم{شرایط نهایی:}
حذف شدن سفارش

\مهم{روند جایگزین:}
ندارد

\newpage

\مهم{مورد کاربرد:}
مشاهده اطلاعات سفارش

\مهم{شماره:}
۲۴

\مهم{عامل اصلی:}

مدیر سامانه

\مهم{عامل فرعی:}
ندارد

\مهم{شرایط اولیه:}

مدیر سامانه وارد سامانه شده باشد.

سفارشی در سامانه وجود داشته باشد.

\مهم{روند اصلی:}

۱. این مورد کاربرد هنگامی شروع می‌شود که مدیر سامانه بخواهد اطلاعات سفارش را مشاهده کند.

۲. یکی از سفارش‌ها را انتخاب می‌کند.

۳. اطلاعات سفارش نمایش داده می‌شود.

\مهم{شرایط نهایی:}
اطلاعات سفارش نمایش داده شود.

\مهم{روند جایگزین:}
ندارد

\newpage

\مهم{مورد کاربرد:}
مشاهده رتبه‌بندی رانندگان

\مهم{شماره:}
۲۵

\مهم{عامل اصلی:}
مدیر سامانه

\مهم{عامل فرعی:}
ندارد

\مهم{شرایط اولیه:}

مدیر سامانه وارد سامانه شده باشد.

راننده‌ای وجود داشته باشد.

باری به مقصد رسیده باشد.

\مهم{روند اصلی:}

۱. این مورد کاربرد هنگامی شروع می‌شود که مدیر سامانه بخواهد رتبه‌بندی رانندگان را مشاهده کند.

۲. مدیر سامانه مشاهده‌ی رتبه‌بندی رانندگان را برمی‌گزیند.

۳. رتبه‌بندی رانندگان برای او نمایش داده می‌شوند.

\مهم{شرایط نهایی:}
رتبه‌بندی رانندگان نمایش داده شود.

\مهم{روند جایگزین:}
ندارد

\newpage

\مهم{مورد کاربرد:}
مشاهده اطلاعات بار

\مهم{شماره:}
۲۶

\مهم{عامل اصلی:}
صاحب بار

\مهم{عامل فرعی:}
ندارد

\مهم{شرایط اولیه:}

صاحب بار وارد سامانه شده باشد.

باری برای صاحب بار در سامانه وجود داشته باشد.

\مهم{روند اصلی:}

۱. این مورد کاربرد هنگامی شروع می‌شود که صاحب بار بخواهد به اطلاعات بار دسترسی پیدا کند.

۲. صاحب بار، بار مد نظر خود را انتخاب می‌کند. 

۳. اطلاعات مربوط به بار برای صاحب بار نمایش داده می‌شود.

\مهم{شرایط نهایی:}
نمایش دادن اطلاعات بار به صاحب بار

\مهم{روند جایگزین:}
ندارد

\newpage

\مهم{مورد کاربرد:}
تایید تحویل بار

\مهم{شماره:}
۲۷

\مهم{عامل اصلی:}
صاحب بار

\مهم{عامل فرعی:}
ندارد

\مهم{شرایط اولیه:}

صاحب بار وارد سامانه شده باشد.

صاحب بار، بار را تحویل گرفته باشد.

\مهم{روند اصلی:}

۱. شامل (مشاهده کردن اطلاعات بار)

۲. صاحب بار، تحویل بار را تایید می‌کند.

۳. تایید تحویل بار در سامانه ثبت می‌گردد.

\مهم{شرایط نهایی:}
تایید و ثبت شدن تحویل بار در سامانه

\مهم{روند جایگزین:}
ندارد

\newpage

\مهم{مورد کاربرد:}
مشاهده موقعیت جغرافیایی بار

\مهم{شماره:}
۲۸

\مهم{عامل اصلی:}
صاحب بار

\مهم{عامل فرعی:}
ندارد

\مهم{شرایط اولیه:}
صاحب بار وارد سامانه شده باشد.

\مهم{روند اصلی:}

۱. شامل(مشاهده کردن اطلاعات بار)

۲. صاحب بار مشاهده‌ی موقعیت بار را برمی‌گزیند.

۳. طول و عرض جغرافیایی و موقعیت مکانی بار نمایش داده می‌شود.

\مهم{شرایط نهایی:}
موقعیت جغرافیایی بار روی نقشه مشخص باشد.

\مهم{روند جایگزین:}
ندارد

\newpage

\مهم{مورد کاربرد:}
مشاهده کردن اطلاعات بار

\مهم{شماره:}
۲۹

\مهم{عامل اصلی:}
صاحب بار

\مهم{عامل فرعی:}
ندارد

\مهم{شرایط اولیه:}
صاحب بار وارد سامانه شده باشد.

صاحب بار، باری در سامانه داشته باشد.

\مهم{روند اصلی:}

۱. این مورد کاربرد هنگامی شروع می‌شود که صاحب بار بخواهد اطلاعات به مربوط بار را مشاهده کند.

۲. مدیر سامانه یکی از بارها را انتخاب می‌کند.

۳. اطلاعات مربوط به بار نمایش داده می‌شود.

\مهم{شرایط نهایی:}
اطلاعات مربوط به بار نمایش داده شود.

\مهم{روند جایگزین:}
ندارد

\newpage

\مهم{مورد کاربرد:}
ثبت امتیاز

\مهم{شماره:}
۳۰

\مهم{عامل اصلی:}
صاحب بار

\مهم{عامل فرعی:}
ندارد

\مهم{شرایط اولیه:}

صاحب بار وارد سامانه شده باشد.

تحویل بار را تایید کرده باشد.

\مهم{روند اصلی:}

 ۱. این مورد کاربرد هنگامی شروع می‌شود که صاحب بار بخواهد به راننده‌ای که بار را تحویل داده امتیاز دهد.

۲. صاحب بار امتیازی به صاحب بار می‌دهد.

۳. امتیاز را در سامانه ثبت می‌کند.

\مهم{شرایط نهایی:}
ثبت شدن امتیاز در سامانه

\مهم{روند جایگزین:}
ندارد
