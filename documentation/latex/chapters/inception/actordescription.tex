\قسمت{توضیح کنش‌گر‌ها}
در این قسمت به‌صورت خلاصه وظیفه‌ی هر یک از  کنش‌گرهای سامانه توضیح داده می‌شود.

\مهم{۱) راننده:} به کسی اطلاق می‌شود که وظیفه‌ی حمل‌ونقل بار، رساندن آن به مقصد و تحویل به مشتری را دارد. در صورت رخداد حادثه‌ای برای خودروی حمل بار می‌تواند آن را گزارش کند و به اطلاع مدیر سامانه برساند.

\مهم{۲) مدیر سامانه:}
به کسی اطلاق می‌شود که نقش اصلی سامانه را در سامانه ایفا می‌کند. در واقع وظیفه‌های متعددی دارد که ثبت سفارش، اختصاص بار به راننده، اختصاص خودروی حمل بار به بار و... جزو مهم‌ترین وظایف او است.

\مهم{۳) مدیر احراز هویت:}
به کسی اطلاق می‌شود که وظیفه‌ی تایید و تطابق اطلاعاتی که رانندگان هنگام ثبت‌نام وارد کرده‌اند را با اطلاعات حقیقی آن‌ها دارد.

\مهم{۴) صاحب بار:}
به کسی اطلاق می‌شود که صاحب بار و محموله‌ای است که توسط مدیر سامانه در قسمت سفارشات ثبت شده است. این شخص می‌تواند اطلاعات و موقعیت لحظه‌ای بار را ببیند و پس از دریافت آن از راننده، به او امتیاز دهد.

\مهم{۵) زمان:} به کنش‌گری اطلاق می‌شود که در بازه‌های مختلف موقعیت جغرافیایی را به‌روزرسانی می‌کند.
