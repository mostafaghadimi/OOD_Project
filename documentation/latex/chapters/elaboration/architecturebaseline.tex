\قسمت{توضیح \لر{Architectural Baseline}}

برای پیاده‌سازی نیازمندی‌ها و موارد کاربرد  که در فاز \لر{Inception} تعریف شده و در حال تکمیل آن هستیم و هم‌چنین با توجه به محدودیت‌هایی نظیر هزینه‌ی زمانی، هزینه‌ی مالی و ... که با آن روبه‌رو هستیم، مشخص کردن یک ساختار و معماری مناسب، ثابت و قابل اطمینان در این مرحله برای کل پروژه، امر بسیار ضروری و مهمی است.

\مهم{مبنا و اساس معماری پروژه:}

\شروع{شمارش}
	\فقره \مهم{پایگاه‌داده:} از آن‌حایی که داده‌های مورد استفاده در این پروژه دارای ساختار ثابت، قانون‌مند و مشخصی هستند، بنابراین استفاده از پایگاه‌داده‌های \لر{SQL-based} بسیار منطقی است. در حالت کلی پایگاه‌داده‌های \لر{SQL} و \لر{NoSQL} برتری نسبت به دیگری ندارند.
	
	\فقره \مهم{برنامه موبایل:} طراحی و پیاده‌سازی برنامه‌ی موبایل به زبان‌های مختلف مثل اندروید، کاتلین و ... شامل پیچیدگی فنی، هزینه‌ی زمانی و مالی زیاد است و نیروی انسانی پرتعدادی برای این کار نیاز دارد. برای همین، انتخاب ما برای پیاده‌سازی برنامه‌ی موبایل، تکنولوژی \لر{PWA} (مخفف عبارت \لر{"Progressive Web Application"}\زیرنویس{به معنی برنامه‌های تحت وب پیشرفته}) که به تازگی توسط گوگل معرفی شده و بسیار پرکاربرد و با قابلیت‌های فراوان نظیر \لر{cache}، ارسال اعلان\زیرنویس{\لر{notification}}، کار کردن در پیش‌زمینه\زیرنویس{\لر{background}} و ... است.
	
	\فقره \مهم{تکنولوژی \لر{Back-End}:} فریم‌ورک متن‌باز \لر{Django} که بر پایه‌ی زبان پایتون شکل گرفته، دارای مزیت‌های بسیار زیادی است  و توصیه‌ی اکید بر اصل «اختراع نکردن دوباره‌ی چرخ» با مفهوم جلوگیری از کارهای تکراری دارد که ذکر چند مورد از آن‌ها خالی از لطف نیست:
	\شروع{فقرات}
		\فقره بسته‌ی احراز هویت
		\فقره جلوگیری از برخی حفره‌های امنیتی متداول
		\فقره پنل ادمین پیش‌فرض
		\فقره و ...
	\پایان{فقرات}
	روند یادگیری ساده‌ای دارد و برای پروژه که در آن با محدودیت زمانی مواجه هستیم، گزینه‌ی بسیار مناسب و ایده‌آلی است.
	
	برای طراحی و پیاده‌سازی \لر{API}، از تکنولوژی \لر{GraphQL} به جای روش سنتی \لر{Restful API} استفاده خواهیم کرد. از مهم‌ترین مزایای \لر{GraphQL} نسبت به \لر{Restful API} می‌توان به موارد زیر اشاره کرد:
	\شروع{فقرات}
		\فقره انعطاف‌پذیری بالا: داده‌ها دقیقا مطابق با نیاز کاربر برای او ارسال می‌شود.
		\فقره سرعت توسعه بالا و تغییرپذیری: ایجاد تغییر در \لر{Restful API} بسیار زمان‌بر است و ممکن است مجیور باشیم برای توسعه در یکی از بخش‌های \لر{Front-End} یا \لر{Back-End} را متوقف کنیم، در حالی که \لر{GraphQL} این‌طور نیست.
		\فقره کاهش بار سرور: تعداد درخواست‌ها در \لر{GraphQL} نسبت به \لر{Restful API} بسیار کم‌تر است؛ به همین دلیل تعامل با سرور کم‌تر، داده‌های مورد نیاز و نه اضافی استخراج شده و به کاربر ارسال می‌شود.
		\فقره و ...
	\پایان{فقرات}
	\فقره \مهم{تکنولوژی \لر{Front-End}}: مهم‌ترین دلیل انتخاب \لر{React} در این پروژه تسلط اعضای پروژه بر آن است. علاوه‌بر این \لر{React} مزایای زیاد دیگری دارد:
	\شروع{فقرات}
		\فقره استفاده مجدد از \لر{component}ها
		\فقره جامعه بزرگ توسعه‌دهندگان
		\فقره سرعت توسعه‌ی بالا
		\فقره و ...
	\پایان{فقرات}
\پایان{شمارش}