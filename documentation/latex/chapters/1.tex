\فصل{فاز \لر {Inception}}

\قسمت{لیست اولویت‌بندی شده ریسک‌ها}
\noindent \hrulefill

\مهم{شرح ریسک:} کافی نبودن منابع انسانی برای محقق کردن تمامی نیازمندی‌ها

\مهم{محدوده ریسک:} فنی در مراحل طراحی و پیاده‌سازی


\مهم{میزان اهمیت ریسک:} بالا

\مهم{اختمال وقوع ریسک:} بالا. با توجه به گسترده و بزرگ بودن پروژه و پیچیدگی‌های فنی آن، احتمال مواجه دن با خطاهای متعدد در آن بسیار زیاد است و این امر منجر به صرف زمان، هزینه و انرژی می‌شود.

\مهم{راه‌حل پیش‌گیرانه:} برای پیش‌گیری از این مشکل، می‌توان ابتدا نیازمندی‌های از نوع اجباری در مدل \لر{MoSCoW} را پیاده‌سازی کرد و سپس به نیازمندی‌های دیگر پرداخت.

\noindent \hrulefill

\مهم{شرح ریسک:} آشنا نبودن برخی اعضای تیم با تکنولوژی‌های ضروری و در نتیجه زمان‌بر شدن پروسه یادگیری که باعث می‌شود نیازمندی‌ها در زمان‌های پیش‌بینی شده انجام نشوند.

\مهم{محدوده ریسک:} فنی در مراحل طراحی و پیاده‌سازی

\مهم{میزان اهمیت ریسک:} بالا / متوسط

\مهم{اختمال وقوع ریسک:} این ریسک به احتمال قوی اتفاق خواهد افتاد اما تعداد دفعات آن بستگی به گستردگی دانش اعضای تیم دارد.

\مهم{راه‌حل پیش‌گیرانه:} برای پیش‌گیری از مشکلات ناشی از این ریسک می‌توان ددلاین‌ها را کمی جلوتر از زمان واقعی آن‌ها تعریف کرد تا در صورت نیاز به یادگیری، زمان کافی تا ددلاین‌های اصلی وجود داشته باشد.

\noindent \hrulefill

\مهم{شرح ریسک:} سخت بودن کار گروهی و ارتباط بین اعضای تیم در شرایطی که امکان ملاقات حضوری به هیچ‌وجه وجود ندارد. این ریسک باعث می‌شود تا توافق اعضای تیم بر سر موضوعات مهم به‌کندی صورت پذیرد.

\مهم{محدوده ریسک:} فنی و بیش‌تر در مرحله طراحی

\مهم{میزان اهمیت ریسک:} بالا

\مهم{اختمال وقوع ریسک:} این ریسک در حال حاضر وجود دارد و امکان بازگشت به شرایط عادی نیز بعید به نظر می‌رسد.

\مهم{راه‌حل پیش‌گیرانه:} برای حل نسبی مشکل ارتباط، می‌توان از ابزارها و نرم‌افزارهای ارتباطی موجود از جمله تلگرام و یا اسکایپ استفاده کرد.

\noindent \hrulefill


\مهم{شرح ریسک:} احتمال به وجود آمدن سوء تفاهم در مورد جزئیات پروژه با توجه به این‌که صورت پروژه تعریف‌شده خیلی کوتاه بوده و تنها در یک صفحه آورده شده، بالا است. امکان ر نظر گرفته نشدن برخی امکانات و هم‌چنین مبهم بودن امکانات نوشته شده وجود دارد. این ریسک در صورت وقوع می‌تواند باعث تحلیل اشتباه اعضای فنی تیم شده و در فازهای بعدی پروژه وقت زیادی صرف اصلاح و یا حتی پیاده‌سازی دوباره شود.

\مهم{محدوده ریسک:} فنی و در مرحله طراحی و پیاده‌سازی

\مهم{میزان اهمیت ریسک:} بالا

\مهم{اختمال وقوع ریسک:} بالا

\مهم{راه‌حل پیش‌گیرانه:}  برای پیش‌گیری از وقوع این ریسک باید ارتباط با مشتری بیش‌تر شده و با گفت‌وگوی اعضای تیم با مشتریان، ابهامات موجود قبل از پیاده‌سازی و در مرحله طراحی برطرف شود.

\noindent \hrulefill

\newpage

\قسمت{لیست اولویت‌بندی شده نیازمندی‌ها}

در این قسمت لیست اولویت‌بندی شده‌ی نیازمندی‌ها با روش \لر{MoSCoW} آورده شده است. در واقع \لر{MoSCow} میزان اهمیت هر کدام از نیازمندی‌ها را با روش زیر مشخص می‌کند:

۱) \مهم{\لر{M}}\زیرنویس{\لر{Must be implemented}}: نیازمندی‌هایی که حتما باید پیاده‌سازی شوند و در صورت پیاده‌سازی نشدن پروژه با شکست مواجه شده است.

۲) \مهم{\لر{S}}\زیرنویس{\لر{Should be implemented}}: نیازمندی‌هایی که باید پیاده‌سازی شوند و نیازمندی‌های مهمی هستند اما در صورت پیاده‌سازی نشدن اختلالی در پروژه به وجود نمی‌آید.

۳) \مهم{\لر{C}}\زیرنویس{\لر{Could be implemented}}: نیازمندی‌هایی که در صورت وجود وقت و زمان کافی بهتر است پیاده‌سازی شوند.

۴) \مهم{\لر{W}}\زیرنویس{\لر{Won't be implemented}}: نیازمندی‌هایی که نباید پیاده‌سازی شوند.

\شروع{فقرات}
	\فقره \مهم{بخش امور کاربری}
	\شروع{فقرات}
		\فقره افزودن مدیر احراز هویت (\لر{M})
		\فقره افزودن مدیر سامانه (\لر{M})
		\فقره ثبت‌نام کردن صاحب بار (\لر{M})
		\فقره ثبت‌نام کردن راننده
		\فقره وارد شدن مدیر سامانه
		\فقره وارد شدن راننده در برنامه موبایل
		\فقره وارد شدن مدیر احراز هویت
		\فقره وارد شدن راننده در سایت
		\فقره وارد شدن صاحب بار
		\فقره خروج از سامانه
	\پایان{فقرات}
	\فقره \مهم{بخش راننده}
	\شروع{فقرات}
		\فقره مشاهده تاریخچه بار
		\فقره مشاهده اطلاعات بار
		\فقره چاپ کردن بارنامه
		\فقره اعلام کردن حادثه
	\پایان{فقرات}
	
	\فقره \مهم{بخش زمان}
	\شروع{فقرات}
		\فقره به‌روزرسانی موقعیت جغرافیایی بار
	\پایان{فقرات}
	\فقره \مهم{بخش مدیریت}
	\شروع{فقرات}
		\فقره تایید کردن اطلاعات راننده
		\فقره تخصیص خودرو حمل بار به بار
		\فقره اختصاص بار به راننده
		\فقره ویرایش اطلاعات راننده
		\فقره تعیین وضعیت راننده		
		\فقره مشاهده موقعیت جغرافیایی بار
		\فقره ثبت خودرو حمل بار جدید
		\فقره مشاهده وضعیت خودرو حمل بار
		\فقره مشاهده کردن اطلاعات بار
		\فقره مشاهده کردن اطلاعات خودرو حمل بار
		\فقره مشاهده کردن اطلاعات راننده
		\فقره ثبت کردن سفارش و اختصاص کد
		\فقره حذف کردن سفارش
		\فقره ویرایش کرن سفارش
		\فقره مشاهده اطلاعات سفارش
		\فقره مشاهده رتبه‌بندی رانندگان
		
	\پایان{فقرات}
	\فقره \مهم{بخش سفارش}
	\شروع{فقرات}
		\فقره مشاهده تاریخچه سفارش‌ها
		\فقره مشاهده اطلاعات سفارش
		\فقره تایید تحویل بار
		\فقره مشاهده موقعیت جغرافیایی بار
		\فقره مشاهده کردن اطلاعات بار
		\فقره ثبت امتیاز
	\پایان{فقرات}
\پایان{فقرات}

\newpage

\قسمت{نمودار مورد کاربرد}

نمودار مورد کاربرد برای بیان و مدل‌سازی روابط بین نیازمندی‌ها و ارتباط آن‌ها با بیرون سامانه‌ی مدیریت حمل‌ونقل بار بین شهری استفاده می‌شود. این سامانه برای مدیریت بهتر منابع یک شرکت بازرگانی فعال در حوزه‌ی حمل‌ونقل بین شهری ایجاد شده است. هدف از ایجاد چنین سامانه‌ای، این است که با در اختیار داشتن اطلاعات لحظه‌ای بار و سفارشات، از لحظه شروع بارگیری تا تحویل آن در مقصد، شفافیت و بهره‌وری کار را بهبود بخشد. 

در این پروژه قرار است تا یک سامانه تحت وب و یک برنامه موبایل پیاده‌سازی شود. در سامانه تحت وب باید اطلاعاتی از قبیل اطلاعات راننده‌ها، خودروهای حمل‌ونقل بار و کالا، اطلاعات بار، وضعیت سفارش و دیگر مواردی که در ادامه  به‌طور کامل‌تر و با جزئیات گفته می‌شود، به شکل متمرکز قرار گرفته باشد. در برنامه‌ی موبایل باید موقعیت بار در طول مسیر برای صاحب بار و مدیر سامانه مشخص باشد.


\مهم{نمودار مورد کاربرد بخش امور کاربری:}

\شروع{شکل}[hb]
\centerimg{./panel.png}{\linewidth}
\شرح{نمودار مورد کاربرد بخش امور کاربری}
\برچسب{شکل:نمودار مورد کاربرد بخش امور کاربری}
\پایان{شکل}


\مهم{نمودار مورد کاربرد بخش رانندگان:}

\شروع{شکل}[hb]
\centerimg{./driver.png}{\linewidth}
\شرح{نمودار مورد کاربرد بخش رانندگان}
\برچسب{شکل:نمودار مورد کاربرد بخش رانندگان}
\پایان{شکل}


\مهم{نمودار مورد کاربرد بخش مدیریت سامانه:}

\شروع{شکل}[hb]
\centerimg{./admin.png}{\linewidth}
\شرح{نمودار مورد کاربرد بخش مدیریت سامانه}
\برچسب{شکل: نمودار مورد کاربرد بخش مدیریت سامانه:}
\پایان{شکل}

\مهم{نمودار مورد کاربرد بخش گزارش‌گیری سامانه:}

\شروع{شکل}[hb]
\centerimg{./time.png}{\linewidth}
\شرح{نمودار مورد کاربرد بخش گزارش‌گیری سامانه}
\برچسب{شکل:نمودار مورد کاربرد بخش گزارش‌گیری سامانه}
\پایان{شکل}

\مهم{نمودار مورد کاربرد بخش سفارش:}

\شروع{شکل}[hb]
\centerimg{./order.png}{\linewidth}
\شرح{نمودار مورد کاربرد بخش سفارش}
\برچسب{شکل:نمودار مورد کاربرد بخش سفارش}
\پایان{شکل}

\newpage
\قسمت{توضیح کنش‌گر‌ها}
در این قسمت به‌صورت خلاصه وظیفه‌ی هر یک از  کنش‌گرهای سامانه توضیح داده می‌شود.

\مهم{۱) راننده:} به کسی اطلاق می‌شود که وظیفه‌ی حمل‌ونقل بار، رساندن آن به مقصد و تحویل به مشتری را دارد. در صورت رخداد حادثه‌ای برای خودروی حمل بار می‌تواند آن را گزارش کند و به اطلاع مدیر سامانه برساند.

\مهم{۲) مدیر سامانه:}
به کسی اطلاق می‌شود که نقش اصلی سامانه را در سامانه ایفا می‌کند. در واقع وظیفه‌های متعددی دارد که ثبت سفارش، اختصاص بار به راننده، اختصاص خودروی حمل بار به بار و... جزو مهم‌ترین وظایف او است.

\مهم{۳) مدیر احراز هویت:}
به کسی اطلاق می‌شود که وظیفه‌ی تایید و تطابق اطلاعاتی که رانندگان هنگام ثبت‌نام وارد کرده‌اند را با اطلاعات حقیقی آن‌ها دارد.

\مهم{۴) صاحب بار:}
به کسی اطلاق می‌شود که صاحب بار و محموله‌ای است که توسط مدیر سامانه در قسمت سفارشات ثبت شده است. این شخص می‌تواند اطلاعات و موقعیت لحظه‌ای بار را ببیند و پس از دریافت آن از راننده، به او امتیاز دهد.

\مهم{۵) زمان:}

\newpage
\قسمت{مشخصات موارد کاربرد}

 در این قسمت لیست توضیحات، نحوه شروع و  روندهای جایگزین موارد کاربرد آورده شده است.

\مهم{مورد کاربرد:}
افزودن مدیر احراز هویت

\مهم{شماره:}
۱

\مهم{عامل اصلی:}
مدیر سامانه

\مهم{عامل فرعی:}
ندارد

\مهم{شرایط اولیه:}
مدیر سامانه، وارد سامانه شده باشد.

\مهم{روند اصلی:}

۱. این مورد کاربرد وقتی آغاز می‌شود که مدیر سامانه بخواهد یک مدیر احراز هویت جدید به سامانه اضافه کند.

۲. مدیر سامانه اطلاعات مربوط به مدیر احراز هویت جدید را وارد می‌کند. این اطلاعات شامل نام، نام‌خانوادگی، شناسه‌کاربری و  رمز عبور می‌باشد.

۳. مدیر سامانه اطلاعات مدیر احراز هویت جدید را در سامانه ثبت می‌کند.

\مهم{شرایط نهایی:}
افزوده شدن مدیر احراز هویت جدید

\مهم{روند جایگزین:}
ندارد

\noindent \hrulefill

\مهم{مورد کاربرد:}
افزودن مدیر سامانه

\مهم{شماره:}
۲
\مهم{عامل اصلی:}
مدیر سامانه

\مهم{عامل فرعی:}
ندارد

\مهم{شرایط اولیه:}
مدیر سامانه، وارد سامانه شده باشد.

\مهم{روند اصلی:}

۱. این مورد کاربرد وقتی آغاز می‌شود که مدیر سامانه بخواهد یک مدیر سامانه‌ی جدید ایجاد کند.

۲. مدیر سامانه اطلاعات مدیر سامانه‌ی جدید را وارد می‌کند. این اطلاعات شامل نام، نام‌خانوادگی، شناسه کاربری، رمز عبور و شماره تماس می‌باشد.

۳. مدیر سامانه اطلاعات مدیر سامانه جدید را ثبت می‌کند.

\مهم{شرایط نهایی:}
افزوده شدن مدیر سامانه جدید

\مهم{روند جایگزین:}
ندارد


\noindent \hrulefill

\مهم{مورد کاربرد:}
ثبت‌نام کردن صاحب بار

\مهم{شماره:}
۳

\مهم{عامل اصلی:}
صاحب بار

\مهم{عامل فرعی:}
ندارد

\مهم{شرایط اولیه:}
ندارد

\مهم{روند اصلی:}

۱. این مورد کاربرد زمانی شروع می‌شود که بازدیدکننده سایت بخواهد در سامانه به عنوان صاحب بار ثبت‌نام کند.

۲. صاحب بار اطلاعات مربوط به خودش را در سامانه وارد می‌کند. این اطلاعات شامل نام، نام‌خانوادگی، شناسه کاربری، رمز عبور و شماره تماس می‌باشد.

۳. صاحب بار اطلاعاتش را در سامانه ثبت می‌کند.

\مهم{شرایط نهایی:}
افزوده شدن یک صاحب بار جدید

\مهم{روند جایگزین:}
ندارد

\noindent \hrulefill

\مهم{مورد کاربرد:}
ثبت‌نام کردن راننده

\مهم{شماره:}
۴

\مهم{عامل اصلی:}
راننده

\مهم{عامل فرعی:}
ندارد

\مهم{شرایط اولیه:}
ندارد

\مهم{روند اصلی:}

۱. این مورد کاربرد هنگامی شروع می‌شود که بازدید کننده سایت بخواهد به عنوان راننده در سامانه ثبت‌نام کند.

۲. راننده اطلاعات خود را در سامانه وارد می‌کند. این اطلاعات شامل نام، نام خانوادگی، شماره تماس، شماره شناسنامه و عکس پرسنلی است.

۳. راننده اطلاعات خود را در سامانه ثبت می‌کند.

\مهم{شرایط نهایی:}
ثبت اطلاعات متقاضی برای نقش راننده

\مهم{روند جایگزین:}
ندارد

\noindent \hrulefill

\مهم{مورد کاربرد:}
وارد شدن مدیر سامانه

\مهم{شماره:}
۵

\مهم{عامل اصلی:}
مدیر سامانه

\مهم{عامل فرعی:}
ندارد

\مهم{شرایط اولیه:}
ندارد

\مهم{روند اصلی:}

۱. این مورد کاربرد زمانی شروع می‌شود که مدیر سامانه بخواهد وارد سامانه شود.

۲. مدیر سامانه اطلاعات کاربری خود را وارد می‌کند. اطلاعات کاربری شامل شناسه کاربری و رمز عبور می‌باشد.

۳. مدیر سامانه درخواست ورود به سایت را می‌دهد.

\مهم{شرایط نهایی:}
ورود مدیر سامانه به بخش مدیریت

\مهم{روند جایگزین:}
عدم صحت اطلاعات

\noindent \hrulefill

\مهم{مورد کاربرد:}
روند جایگزین مورد کاربرد: وارد شدن مدیر سامانه: عدم صحت اطلاعات

\مهم{شماره:}
۵.۱

\مهم{عامل اصلی:}
مدیر سامانه

\مهم{عامل فرعی:}
ندارد

\مهم{شرایط اولیه:}
شناسه کاربری یا رمز عبور اشتباه وارد شده باشد.

\مهم{روند اصلی:}

۱. این مورد کاربرد هنگامی شروع می‌شود که شناسه کاربری یا رمز عبور اشتباه وارد شده باشد.

۲. پیغامی مبنی بر عدم صحت اطلاعات به مدیر سامانه نمایش داده می‌شود.

\مهم{شرایط نهایی:}
ندارد

\مهم{روند جایگزین:}
ندارد

\مهم{نکته:} به دلیل شباهت بسیار زیاد و هم‌چنین واضح بودن دیگر موارد کاربرد وارد شدن به سامانه و هم‌چنین روند جایگزین آن‌ها از آوردن آن‌ها در این قسمت اجتناب می‌کنیم.

\noindent \hrulefill

\مهم{مورد کاربرد:}
خروج از سامانه

\مهم{شماره:}
۶

\مهم{عامل اصلی:}
کاربر سایت

\مهم{عامل فرعی:}
ندارد

\مهم{شرایط اولیه:}
کاربر سایت وارد سامانه شده باشد

\مهم{روند اصلی:}

۱. این مورد کاربرد هنگامی فعال می‌شود که کاربر سایت بخواهد از سیستم خارج شود.

۲. کاربر سایت درخواست خروج را ارسال می‌کند.

\مهم{شرایط نهایی:}
خروج کاربر سایت از سامانه 

\مهم{روند جایگزین:}
ندارد

\noindent \hrulefill

\مهم{مورد کاربرد:}
مشاهده اطلاعات بارنامه

\مهم{شماره:}
۷

\مهم{عامل اصلی:}
راننده

\مهم{عامل فرعی:}
ندارد

\مهم{شرایط اولیه:}
راننده وارد سامانه شده باشد.

\مهم{روند اصلی:}

۱. این مورد کاربرد هنگامی شروع می‌شود که راننده بخواهد اطلاعات مربوط به بار را مشاهده کند.

۲. راننده باری را انتخاب می‌کند و درخواستی برای مشاهده اطلاعات ارسال می‌شود.

۳. اطلاعات مربوط به بار به راننده نمایش داده می‌شود.

\مهم{شرایط نهایی:}
نمایش اطلاعات بار به راننده

\مهم{روند جایگزین:}
ندارد

\noindent \hrulefill

\مهم{مورد کاربرد:}
چاپ کردن اطلاعات بارنامه

\مهم{شماره:}
۷

\مهم{عامل اصلی:}
راننده

\مهم{عامل فرعی:}
ندارد

\مهم{شرایط اولیه:}
راننده وارد سامانه شده باشد.

\مهم{روند اصلی:}

۱. این مورد کاربرد هنگامی شروع می‌شود که راننده بخواهد اطلاعات مربوط به یک بارنامه را چاپ کند. 

۲. راننده یک بار (بار موجود) را انتخاب کرده و درخواست مشاهده‌ی اطلاعات آن را ارسال می‌کند.

۳. اطلاعات مربوط به بار به راننده نمایش داده می‌شود.

۴. راننده این اطلاعات را می‌تواند به چاپ کند.

\مهم{شرایط نهایی:}
چاپ شدن اطلاعات بارنامه

\مهم{روند جایگزین:}
ندارد

\noindent \hrulefill

\مهم{مورد کاربرد:}
اعلام کردن حادثه

\مهم{شماره:}
۸

\مهم{عامل اصلی:}
راننده

\مهم{عامل فرعی:}
ندارد

\مهم{شرایط اولیه:}

راننده وارد سامانه شده باشد.

حادثه‌ای در خلال حمل‌و‌نقل بار برای راننده و یا خودروی حمل‌ بار اتفاق افتاده باشد.

\مهم{روند اصلی:}

۱. این مورد کاربرد هنگامی فعال می‌شود که برای راننده و یا خودروی حمل بار اتفاقی رخ داده باشد و  راننده بخواهد آن را گزارش کند.

۲. راننده اعلام حادثه را انتخاب می‌کند.

۳. شرح حادثه و محل وقوع آن را وارد می‌کند.

۴. با ثبت کردن موارد مذکور، حادثه را در سامانه ثبت می‌کند.

\مهم{شرایط نهایی:}
اعلام کردن حادثه پیش‌آمده

\مهم{روند جایگزین:}
ندارد

\noindent \hrulefill

\مهم{مورد کاربرد:}
به‌روزرسانی موقعیت جغرافیایی بار

.\مهم{شماره:}
۹

\مهم{عامل اصلی:}
زمان

\مهم{عامل فرعی:}
ندارد

\مهم{شرایط اولیه:}
ندارد

\مهم{روند اصلی:}

۱. این مورد کاربرد در بازه‌های زمانی مشخصی شروع به کار می‌کند.

۲. موقعیت جغرافیایی بار نظیر طول و عرض جغرافیایی به‌روز می‌شوند.

\مهم{شرایط نهایی:}
به‌روز شدن مؤلفه‌های موقعیت جغرافیایی	

\مهم{روند جایگزین:}
ندارد

\noindent \hrulefill

\مهم{مورد کاربرد:}
تایید کردن اطلاعات راننده

.\مهم{شماره:}
۱۰

\مهم{عامل اصلی:}
مدیر احراز هویت

\مهم{عامل فرعی:}
ندارد

\مهم{شرایط اولیه:}
مدیر احراز هویت وارد سامانه شده باشد.

\مهم{روند اصلی:}

۱. این مورد کاربرد هنگامی شروع به کار می‌کند که مدیر احراز هویت یکی از متقاضیان رانندگی در سامانه را انتخاب می‌کند.

۲. مدیر احراز هویت اطلاعات وارد شده توسط متقاضی رانندگی در سامانه را به دور دقیق چک می‌کند.

۲.۱. اگر اطلاعات وارد شده مطابق با اطلاعات حقیقی راننده باشد، هویت او را تایید کرده و به عنوان راننده تایید می‌شود و یک شناسه کاربری و رمز عبور به او اختصاص داده می‌شود.

۲.۲ اگر اطلاعات وارد شده درست نباشد، آن‌گاه درخواست او برای رانندگی در سامانه حذف می‌شود.

\مهم{شرایط نهایی:}
تایید کردن صحت اطلاعات راننده

\مهم{روند جایگزین:}
ندارد

\noindent \hrulefill

\مهم{مورد کاربرد:}
تخصیص خودرو حمل بار به بار

.\مهم{شماره:}
۱۱

\مهم{عامل اصلی:}
مدیر سامانه

\مهم{عامل فرعی:}
ندارد

\مهم{شرایط اولیه:}

مدیر سامانه وارد سامانه شده باشد. 

خودروی حمل‌باری در گاراژ موجود باشد.

سفارشی در سیستم موجود باشد.

\مهم{روند اصلی:}

۱. این مورد کاربرد هنگامی شروع می‌شود که مدیرسامانه بخواهد به یکی از سفارش‌ها ماشینی اختصاص دهد.

۲. مدیر سامانه ابتدا یکی از سفارش‌ها را انتخاب می‌کند.

۳. سپس یک خودروی حمل بار موجود در گاراژ را به آن اختصاص می‌دهد.

۴. سفارش را تبدیل به بار می‌کند.

\مهم{شرایط نهایی:}
اختصاص یافتن بار به خودروی حمل بار

\مهم{روند جایگزین:}
ندارد

\noindent \hrulefill

\مهم{مورد کاربرد:}
ویرایش اطلاعات راننده

.\مهم{شماره:}
۱۲

\مهم{عامل اصلی:}
مدیر سامانه

\مهم{عامل فرعی:}
ندارد

\مهم{شرایط اولیه:}
مدیر سامانه وارد سامانه شده باشد.

\مهم{روند اصلی:}

۱. شامل(مشاهده کردن اطلاعات راننده)

۲. مدیر سامانه ویرایش را برمی‌گزیند.

۳. اطلاعات که می‌خواهد ویرایش کند را وارد می‌کند.

۴. تغییرات را ذخیره می‌کند.

\مهم{شرایط نهایی:}
اطلاعات راننده ویرایش شود

\مهم{روند جایگزین:}
ندارد

\noindent \hrulefill

\مهم{مورد کاربرد:}
اختصاص دادن بار به راننده

.\مهم{شماره:}
۱۳

\مهم{عامل اصلی:}
مدیر سامانه

\مهم{عامل فرعی:}
ندارد

\مهم{شرایط اولیه:}

مدیر سامانه وارد سامانه شده باشد.

باری موجود باشد.

راننده‌ای در حالت آزاد وجود داشته باشد.

\مهم{روند اصلی:}

۱. این مورد کاربرد هنگامی شروع می‌شود که مدیر سامانه بخواهد به راننده‌ای، بار اختصاص دهد.

۲. مدیر سامانه یکی از بارها را انتخاب می‌کند.

۳. برای بار انتخاب شده، یکی از راننده‌ها را انتخاب می‌کند.

\مهم{شرایط نهایی:}
بار به راننده اختصاص یابد

\مهم{روند جایگزین:}
ندارد

\noindent \hrulefill

\مهم{مورد کاربرد:}
تعیین وضعیت راننده

.\مهم{شماره:}
۱۴

\مهم{عامل اصلی:}
مدیر سامانه

\مهم{عامل فرعی:}
ندارد

\مهم{شرایط اولیه:}
مدیر سامانه وارد سامانه شده باشد.

\مهم{روند اصلی:}

۱. شامل (مشاهده کردن اطلاعات راننده)

۱.۱ اگر حادثه‌ای توسط راننده گزارش شده باشد یا مدیر سامانه به راننده بار تخصیص بدهد و یا صاحب بار وضعیت بار را مشخص کند، مدیر سامانه وضعیت راننده را تغییر می‌دهد.

\مهم{شرایط نهایی:}
وضعیت راننده تغییر یابد.

\مهم{روند جایگزین:}
ندارد


\noindent \hrulefill

\مهم{مورد کاربرد:}
مشاهده موقعیت جغرافیایی بار

.\مهم{شماره:}
۱۵

\مهم{عامل اصلی:}
مدیر سامانه

\مهم{عامل فرعی:}
ندارد

\مهم{شرایط اولیه:}
مدیر سامانه وارد سامانه شده باشد.

\مهم{روند اصلی:}

۱. شامل(مشاهده کردن اطلاعات بار)

۲. مدیر سامانه مشاهده‌ی موقعیت بار را برمی‌گزیند.

۳. طول و عرض جغرافیایی و موقعیت مکانی بار نمایش داده می‌شود.

\مهم{شرایط نهایی:}
موقعیت جغرافیایی بار روی نقشه مشخص باشد.

\مهم{روند جایگزین:}
ندارد

\noindent \hrulefill

\مهم{مورد کاربرد:}
ثبت خودرو حمل بار جدید

.\مهم{شماره:}
۱۶

\مهم{عامل اصلی:}
مدیر سامانه

\مهم{عامل فرعی:}
ندارد

\مهم{شرایط اولیه:}
مدیر سامانه وارد سامانه شده باشد.

\مهم{روند اصلی:}

۱. این مورد کاربرد هنگامی شروع می‌شود که مدیر سامانه بخواهد خودرو حمل بار جدیدی را ثبت کند.

۲. به همین منظور ثبت خودروی حمل بار جدید را برمی‌گزیند.

۳. اطلاعات مربوط به خودرو حمل بار را وارد می‌کند.

۴. اطلاعات خودرو حمل بار را ثبت می‌کند.

\مهم{شرایط نهایی:}
خودرو حمل بار جدید در سامانه ثبت شود.

\مهم{روند جایگزین:}
ندارد

\noindent \hrulefill

\مهم{مورد کاربرد:}
مشاهده وضعیت خودرو حمل بار

.\مهم{شماره:}
۱۷

\مهم{عامل اصلی:}
مدیر سامانه

\مهم{عامل فرعی:}
ندارد

\مهم{شرایط اولیه:}
مدیر سامانه وارد سامانه شده باشد.

\مهم{روند اصلی:}

۱. شامل (مشاهده کردن اطلاعات خودرو حمل بار)

۲. مدیر سامانه مشاهده وضعیت خودرو حمل بار را انتخاب می‌کند.

۳. وضعیت خودروی حمل بار نمایش داده می‌شود.

\مهم{شرایط نهایی:}
وضعیت خودرو حمل بار نمایش داده شود.

\مهم{روند جایگزین:}
ندارد

\noindent \hrulefill

\مهم{مورد کاربرد:}
مشاهده کردن اطلاعات بار

.\مهم{شماره:}
۱۸

\مهم{عامل اصلی:}
مدیر سامانه

\مهم{عامل فرعی:}
ندارد

\مهم{شرایط اولیه:}
مدیر سامانه وارد سامانه شده باشد.

\مهم{روند اصلی:}

۱. این مورد کاربرد هنگامی شروع می‌شود که مدیر سامانه بخواهد اطلاعات مربوط بار را مشاهده کند.

۲. مدیر سامانه یکی از بارها را انتخاب می‌کند.

۳. اطلاعات مربوط به بار نمایش داده می‌شود.

\مهم{شرایط نهایی:}
اطلاعات مربوط به بار نمایش داده شود.

\مهم{روند جایگزین:}
ندارد

\noindent \hrulefill

\مهم{مورد کاربرد:}
مشاهده کردن اطلاعات خودرو حمل بار

.\مهم{شماره:}
۱۹

\مهم{عامل اصلی:}
مدیر سامانه

\مهم{عامل فرعی:}
ندارد

\مهم{شرایط اولیه:}
مدیر سامانه وارد سامانه شده باشد.

\مهم{روند اصلی:}

۱. این مورد کاربرد هنگامی شروع می‌شود که مدیر سامانه بخواهد اطلاعات مربوط به خودرو حمل بار را مشاهده کند.

۲. یکی از خودروهای حمل بار را انتخاب می‌کند.

۳. اطلاعات خودروی حمل بار به او نمایش داده می‌شود.

\مهم{شرایط نهایی:}
نمایش اطلاعات خودرو حمل بار

\مهم{روند جایگزین:}
ندارد

\noindent \hrulefill



\مهم{مورد کاربرد:}
مشاهده کردن اطلاعات راننده

.\مهم{شماره:}
۲۰

\مهم{عامل اصلی:}
مدیر سامانه

\مهم{عامل فرعی:}
ندارد

\مهم{شرایط اولیه:}
مدیر سامانه وارد سامانه شده باشد

\مهم{روند اصلی:}

۱. این مورد کاربرد هنگامی شروع می‌شود که مدیر سامانه بخواهد اطلاعات یک راننده را مشاهده کند.

۲. مدیر سامانه یکی از راننده‌ها را انتخاب می‌کند.

۳. اطلاعات مربوط به راننده برای او نمایش داده می‌شود.

\مهم{شرایط نهایی:}
مشاهده تمامی اطلاعات راننده

\مهم{روند جایگزین:}
ندارد

\noindent \hrulefill

\مهم{مورد کاربرد:}
ثبت کردن سفارش جدید

.\مهم{شماره:}
۲۱

\مهم{عامل اصلی:}
مدیر سامانه

\مهم{عامل فرعی:}
ندارد

\مهم{شرایط اولیه:}
مدیر سامانه وارد سامانه شده باشد.

\مهم{روند اصلی:}

۱. این مورد کاربرد زمانی شروع می‌شود که مدیر سامانه بخواهد سفارش جدیدی را ثبت کند.

۲. مدیر سامانه ثبت سفارش جدید را انتخاب می‌کند.

۳. اطلاعات مربوط به سفارش را نظیر مبدا، مقصد، وزن و … را کامل می‌کند.

۴. اطلاعات مربوط به سفارش را ذخیره می‌کند.

\مهم{شرایط نهایی:}
سفارش جدید در سامانه ثبت شود.

\مهم{روند جایگزین:}
ندارد 


\noindent \hrulefill

\مهم{مورد کاربرد:}
ویرایش کردن سفارش

.\مهم{شماره:}
۲۲

\مهم{عامل اصلی:}
مدیر سامانه

\مهم{عامل فرعی:}
ندارد

\مهم{شرایط اولیه:}

مدیر سامانه وارد سامانه شده باشد.

سفارشی در سامانه وجود داشته باشد.

\مهم{روند اصلی:}

۱. شامل (مشاهده اطلاعات سفارش)

۲. اطلاعاتی که باید ویرایش شوند، اضافه می‌شوند.

۳. اطلاعات در سامانه ثبت می‌شود.

\مهم{شرایط نهایی:}
اطلاعات سفارش ویرایش شود.

\مهم{روند جایگزین:}
ندارد

\noindent \hrulefill

\مهم{مورد کاربرد:}
حذف کردن سفارش

.\مهم{شماره:}
۲۳

\مهم{عامل اصلی:}
مدیر سامانه

\مهم{عامل فرعی:}
ندارد

\مهم{شرایط اولیه:}
مدیر سامانه وارد سامانه شود.

سفارشی در سامانه وجود داشته باشد.

\مهم{روند اصلی:}

۱. شامل (مشاهده اطلاعات سفارش)

۲. مدیر سامانه حذف را برمی‌گزیند.

۳. سفارش از سامانه حذف می‌شود.

\مهم{شرایط نهایی:}
حذف شدن سفارش

\مهم{روند جایگزین:}
ندارد

\noindent \hrulefill

\مهم{مورد کاربرد:}
مشاهده اطلاعات سفارش

.\مهم{شماره:}
۲۴
\مهم{عامل اصلی:}

مدیر سامانه

\مهم{عامل فرعی:}
ندارد

\مهم{شرایط اولیه:}

مدیر سامانه وارد سامانه شده باشد.

سفارشی در سامانه وجود داشته باشد.

\مهم{روند اصلی:}

۱. این مورد کاربرد هنگامی شروع می‌شود که مدیر سامانه بخواهد اطلاعات سفارش را مشاهده کند.

۲. یکی از سفارش‌ها را انتخاب می‌کند.

۳. اطلاعات سفارش نمایش داده می‌شود.

\مهم{شرایط نهایی:}
اطلاعات سفارش نمایش داده شود.

\مهم{روند جایگزین:}
ندارد

\noindent \hrulefill

\مهم{مورد کاربرد:}
مشاهده رتبه‌بندی رانندگان

.\مهم{شماره:}
۲۵

\مهم{عامل اصلی:}
مدیر سامانه

\مهم{عامل فرعی:}
ندارد

\مهم{شرایط اولیه:}

مدیر سامانه وارد سامانه شده باشد.

راننده‌ای وجود داشته باشد.

باری به مقصد رسیده باشد.

\مهم{روند اصلی:}

۱. این مورد کاربرد هنگامی شروع می‌شود که مدیر سامانه بخواهد رتبه‌بندی رانندگان را مشاهده کند.

۲. مدیر سامانه مشاهده‌ی رتبه‌بندی رانندگان را برمی‌گزیند.

۳. رتبه‌بندی رانندگان برای او نمایش داده می‌شوند.

\مهم{شرایط نهایی:}
رتبه‌بندی رانندگان نمایش داده شود.

\مهم{روند جایگزین:}
ندارد

\noindent \hrulefill

\مهم{مورد کاربرد:}
مشاهده اطلاعات بار

.\مهم{شماره:}
۲۶

\مهم{عامل اصلی:}
صاحب بار

\مهم{عامل فرعی:}
ندارد

\مهم{شرایط اولیه:}

صاحب بار وارد سامانه شده باشد.

باری برای صاحب بار در سامانه وجود داشته باشد.

\مهم{روند اصلی:}

۱. این مورد کاربرد هنگامی شروع می‌شود که صاحب بار بخواهد به اطلاعات بار دسترسی پیدا کند.

۲. صاحب بار، بار مد نظر خود را انتخاب می‌کند. 

۳. اطلاعات مربوط به بار برای صاحب بار نمایش داده می‌شود.

\مهم{شرایط نهایی:}
نمایش دادن اطلاعات بار به صاحب بار

\مهم{روند جایگزین:}
ندارد

\noindent \hrulefill

\مهم{مورد کاربرد:}
تایید تحویل بار

.\مهم{شماره:}
۲۷

\مهم{عامل اصلی:}
صاحب بار

\مهم{عامل فرعی:}
ندارد

\مهم{شرایط اولیه:}

صاحب بار وارد سامانه شده باشد.

صاحب بار، بار را تحویل گرفته باشد.

\مهم{روند اصلی:}

۱. شامل (مشاهده کردن اطلاعات بار)

۲. صاحب بار، تحویل بار را تایید می‌کند.

۳. تایید تحویل بار در سامانه ثبت می‌گردد.

\مهم{شرایط نهایی:}
تایید و ثبت شدن تحویل بار در سامانه

\مهم{روند جایگزین:}
ندارد

\noindent \hrulefill

\مهم{مورد کاربرد:}
مشاهده موقعیت جغرافیایی بار

.\مهم{شماره:}
۲۸

\مهم{عامل اصلی:}
صاحب بار

\مهم{عامل فرعی:}
ندارد

\مهم{شرایط اولیه:}
صاحب بار وارد سامانه شده باشد.

\مهم{روند اصلی:}

۱. شامل(مشاهده کردن اطلاعات بار)

۲. مدیر سامانه مشاهده‌ی موقعیت بار را برمی‌گزیند.

۳. طول و عرض جغرافیایی و موقعیت مکانی بار نمایش داده می‌شود.

\مهم{شرایط نهایی:}
موقعیت جغرافیایی بار روی نقشه مشخص باشد.

\مهم{روند جایگزین:}
ندارد

\noindent \hrulefill

\مهم{مورد کاربرد:}
مشاهده کردن اطلاعات بار

.\مهم{شماره:}
۲۹

\مهم{عامل اصلی:}
صاحب بار

\مهم{عامل فرعی:}
ندارد

\مهم{شرایط اولیه:}
صاحب بار وارد سامانه شده باشد.

صاحب بار، باری در سامانه داشته باشد.

\مهم{روند اصلی:}

۱. این مورد کاربرد هنگامی شروع می‌شود که صاحب بار بخواهد اطلاعات به مربوط بار را مشاهده کند.

۲. مدیر سامانه یکی از بارها را انتخاب می‌کند.

۳. اطلاعات مربوط به بار نمایش داده می‌شود.

\مهم{شرایط نهایی:}
اطلاعات مربوط به بار نمایش داده شود.
\مهم{روند جایگزین:}
ندارد

\noindent \hrulefill

\مهم{مورد کاربرد:}
ثبت امتیاز

.\مهم{شماره:}
۳۰

\مهم{عامل اصلی:}
صاحب بار

\مهم{عامل فرعی:}
ندارد

\مهم{شرایط اولیه:}

صاحب بار وارد سامانه شده باشد.

تحویل بار را تایید کرده باشد.

\مهم{روند اصلی:}

۱. این مورد کاربرد هنگامی شروع می‌شود که صاحب بار بخواهد به راننده‌ای که بار را تحویل داده امتیاز دهد.

۲. صاحب بار امتیازی به صاحب بار می‌دهد.

۳. امتیاز را در سامانه ثبت می‌کند.

\مهم{شرایط نهایی:}
ثبت شدن امتیاز در سامانه

\مهم{روند جایگزین:}
ندارد

\noindent \hrulefill