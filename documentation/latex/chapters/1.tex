\فصل{مروری بر نمودار \لر{UML}}

\قسمت {مقدمه}مدر این فصل به مرور و آشنایی با نمودارهای زبان مدل‌سازی یک‌پارچه خواهیم پرداخت.




\قسمت{تعریف}

\لر{UML} یا زبان مدل‌سازی یک‌پارچه \زیرنویس{\لر{Unified Modeling Language}}، زبان استاندارد مهندسی نرم‌افزار برای مدل‌سازی است. مجموعه‌ای از بهترین رویه‌های عملی \زیرنویس{practice} را برای مدل کردن سیستم‌های بزرگ و پیچیده، ارائه می‌دهد.

\مهم{کاربرد:} برای مشخص کردن \زیرنویس{specifying}،  مجسم کردن\زیرنویس{visualizing}، ساختن \زیرنویس{constructing} و مستند کردن \زیرنویس{documenting} محصولات \زیرنویس{\لر{artifact} در این‌جا به معنی مصنوع و محصول است.} مهندسی نرم‌افزار و هم‌چنین مدل‌سازی کسب‌وکار از آن استفاده می‌شود.

\مهم{سوال:} در چه صورت می‌توان از \لر{UML} استفاده کرد؟ در صورتی که بتوان برای برای آن سیستم \لر{object} متصور د و رفتار سیستم را بتوان به‌صورت تعامل مجموعه‌ای از اشیاء نمایش داد. 

\قسمت{شکل‌گیری \لر{UML}} این اسلاید، تاثیر زبان‌های مدل‌سازی (بعضی از مستطیل‌های طوسی مانند \لر{C++}) را بر روی \لر{UML} نمایش می‌دهد. 

\مهم{نکته ۱:} روش \لر{CRC} \زیرنویس{\لر{Class Responsibility Collaboration}} یکی از مهم‌ترین روش‌ها برای تشخیص کلاس است. 

\مهم{نکته ۲:} منظور از \لر{Formal Specification} تعریف‌های ریاضی است.

\مهم{نکته ۳:} تاثیرگذارترین متدولوژی‌ها روی شکل‌گیری \لر{UML} عبارتند از: 
\شروع{شمارش}
\فقره \لر{Objectory}
\فقره \لر{Booch}
\فقره \لر{OMT}
\پایان{شمارش}

زیرا پایه‌گذاران این متدولوژی‌ها، \لر{UML} را نیز ابداع کرده‌اند.

\قسمت{انواع نمودارها}

در مهندسی نرم‌افزار انواع نمودارها به سه دسته‌ی زیر تقسیم می‌شوند:
\شروع{شمارش}
\فقره \مهم{ساختاری یا \لر{Structural}}: سیستم از چه اجزایی تشکیل شده است و چه رابطه‌ای با یکدیگر دارند؛ مانند نمودار کلاس. 
\فقره \مهم{رفتاری یا \لر{Behavioral}}: اجزای داخلی چگونه با یک‌دیگر کار می‌کنند و ترتیب انجام کارها به چه شکلی است به عبارت دیگر تقدم-تاخر وجود دارد؛ مانند نمودار \لر{flowchart}
\فقره \مهم{وظیفه‌ای یا \لر{Functional}}: \لر{UML} این نمودار را \مهم{ندارد}. کارها را بدون ترتیب توصیف می‌کند. مثل این‌که سقف را از روی یک سازمان برداریم، کارمندان و ارتباطاتشان را مشاهده کنیم.
\پایان{شمارش}

\مهم{نمودار بسته:} از این نمودار فقط در جریان کاری تحلیل استفاده می‌شود. وظیفه‌اش افراز  کلاس‌ها به تعدادی بسته است.

\مهم{نمودار مورد کاربرد:} در این نمودار مشخص می‌شود که کنش‌گرهای سیستم چه سرویس‌هایی از آن می‌گیرند. هر مورد کاربرد یک \لر{functionality} است. 
علاوه‌بر این رابطه‌ی بین موارد کاربرد را مشاهده می‌کنیم و ترتیب ندارند. پس به همین دلیل حزو نمودارهای رفتاری یا \لر{Behavioral} نیست، اما همان‌طور که گفته شد، نمودار \لر{UML}، نمودارهای وظیفه‌ای را ندارد و به همین علت در آن، جزو نمودارهای وظیفه‌ای محسوب می‌شود.

\مهم{نکته: } به نمودار فعالیت، فلوچارت شی‌گرا می‌گویند.

\مهم{نمودار زمانی:} زمانی که محدودیت زمان پاسخ داریم از این نمودار استفاده می‌کنیم؛ مانند سیستم‌های بی‌درنگ.

\قسمت{نسخه‌های \لر{UML}}
در این قسمت تکامل نسخه‌های مختلف \لر{UML} و مفاهیم جدید آن مورد بررسی قرار گرفته شده است.

\قسمت{نمودار کلاس}
هر کلاس یک مستطیل چند قسمتی\زیرنویس{\لر{multi compartment}} است که عموما سه قسمت است. در جریان کاری تحلیل، وارد جزئیات کلاس‌ها نمی‌شویم؛ مثلا نوع \لر{attribute}ها یا \لر{access visibility} مشخص نمی‌شود.

\مهم{نکته:} هر توارثی \لر{GenSpec} نیست اما هر \لر{GenSpec}، یک رابطه‌ی توارث است. 

\مهم{\لر{Refused Bequest}:} یک ویژگی از پدر به فرزند رسیده است اما فرزند به آن نیازی ندارد. در این حالت چون رابطه‌ی \لر{IS-A} نقض می‌شود، می‌توانیم بگوییم که \لر{GenSpec} نیست. مثلا برای کلاس اسب، متد خوردن و راه رفتن داریم. برای کلاس عنکبوت هم همین دو متد را نیز داریم. اگر عنکبوت را زیرکلاس اسب در نظر بگیریم، یکی از فاجعه بار ترین اشتباهات است؛ زیرا از \لر{inheritance} برای \لر{reuse} استفاده کرده‌ایم. اگر چهار نعل و یورتمه را برای اسب، تعریف کنیم، حالت \لر{Refused Bequest} به وجود می‌آید که مشخص می‌کند رابطه دیگر رابطه‌ی \لر{IS-A} نیست.

\مهم{رابطه‌ی \لر{Association}}: یک رابطه‌ی مانا بین کلاس‌ها برقرار است. در پیاده‌سازی معمولا برای این رابطه یک \لر{attribute} مانند \لر{Foreign Key} تعریف می‌کنیم. اگر این رابطه جهت‌دار باشد، یعنی دید از یک طرف به سمت دیگر وجود ندارد. (مثالی که تو جزوه نوشتم بیارم)

\مهم{رابطه‌ی \لر{Aggregation}}: یک رابطه‌ی \لر{Association}، \لر{Aggregation} است،‌ اگر یکی از سه شرط زیر را داشته باشد (در این رابطه معمولا دید یک‌طرفه از کل به جزء وجود دارد):
\شروع{شمارش}
\فقره تشکیل شدن یا \لر{Assembly}: وقتی یک طرف از \لر{instance}های طرف دیگر تشکیل شده است؛ مانند درخت و برگ‌هایش.

\فقره محتوی بودن یا \لر{Containment}:  وقتی یک طرف \لر{Container} طرف دیگر بشود؛ مانند بسته‌ی پستی و محتویات داخل آن.

\فقره عضویت یا \لر{Membership}: وقتی یک طرف عضو طرف دیگر باشد؛ مانند یک تیم فوتبال و اعضای آن.
\پایان{شمارش}


\مهم{رابطه‌ی \لر{Composition}}: 

یک رابطه‌ی \لر{Aggregation}، \لر{Composition} است، اگر:
\شروع{شمارش}
\فقره \لر{Lifetime dependency} داشته باشیم؛ یعنی با از بین رفتن کل، جزء هم از بین برود.
\فقره \لر{sharing} نداشته باشیم.
\پایان{شمارش}

\مهم{نکته:} با توجه به نکات گفته شده، رابطه‌ی \لر{Assembly} به احتمال زیاد \لر{Composition} است؛ زیرا \لر{Lifetime dependency} دارد.

\مهم{رابطه \لر{Dependency}}: تغییرات در \لر{B} موجب تغییرات در \لر{A} می‌شود؛برای مثال اگر به کلاس \لر{A} پیغام ارسال کنیم و به عنوان پارامتر، نمونه‌ی کلاس \لر{B} را بفرستیم، یک دید موقت به وجود می‌آید. برقراری دید از طریق پارامتر معمولا غیرمانا است.

بسیاری از مدل‌سازها در ابتدای کار، همه‌ی روابط را \لر{dependency} در نظر می‌گیرند و سپس به روابط دیگر تبدیل می‌کنند:
\begin{equation*}
	Dependency \rightarrow Association \rightarrow Aggregation \rightarrow Composition
\end{equation*}
\begin{equation*}
	Dependency \rightarrow GenSpec
\end{equation*}

\قسمت{نمودار \لر{Object}}
این نمودار ساختار \لر{Object}ها را در زمان اجرا نشان می‌دهد. 

\مهم{موارد کاربرد}: ۱) در بعضی از مواقع نمودار کلاس ایجاد ابهام می‌کند. مثلا نمودار کلاس دپارتمان به خودش رابطه‌ی زیردپارتمان دارد که در این اسلاید نمایش داده شده است. ۲) برای بیان وضعیت استثنایی که با نمودار کلاس قابل بیان نیست؛ برای مثال در فوتبال کلاس زمین، بازیکن و ... داریم اما برای بیان وضعیت آفساید به نمودار \لر{Object} نیاز داریم.

