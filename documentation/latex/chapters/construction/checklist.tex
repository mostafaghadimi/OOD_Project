\قسمت{چک‌لیست دست‌آوردها}

این قسمت از دو بخش تشکیل شده است، در مرحله‌ی اول به صورت کیفی کامل بودن نیازمندی‌ها را بررسی کرده و سپس در مرحله‌ی بعدی، جزئیات آن را بررسی خواهیم کرد.

\شروع{فقرات}
 \فقره محقق شدن تمامی موارد کاربرد: عالی
 \فقره کامل بودن کلاس‌ها: عالی
 \فقره \لر{primitive} بودن نمودار: عالی
 \فقره کافی بودن کلاس‌ها: عالی
 \فقره اساس ساختار و معماری برطرف‌کننده نیاز‌ها و ریسک‌های معماری: عالی
 \فقره شناسایی و تشخیص صحیح نیازمندی‌ها: عالی
 \فقره پیاده‌سازی مطابق با کلاس‌های طراحی: عالی
 \فقره بالا بودن \لر{cohesion}: خوب
 \فقره پایین بودن \لر{coupling}: خوب
 \فقره سازگاری نمودارها:‌ عالی
 \فقره کاربرپسند بودن واسط کاربری: عالی
 \فقره درست و کامل بودن \لر{Executable Architecural Baseline}: عالی
 \فقره کامل بودن کارت‌های \لر{CRC}: عالی
\فقره تطابق مدل‌های \لر{back-end} با نمودار کلاس: عالی
\فقره اضافه شدن رابط کاربری به نمودار توالی: عالی
\فقره تکمیل شدن نمودار فعالیت: عالی
\فقره بکارگیری الگوهای طراحی مناسب: عالی
\فقره پیاده‌سازی نیازمندی‌های اصلی در \لر{front-end} و \لر{back-end}: عالی

\پایان{فقرات}
\newpage
\مهم{بررسی کلاس‌ها در پیاده‌سازی}

\شروع{فقرات}
	\فقره \مهم{کلاس \لر{Driver}}:
	
	\شروع{فقرات}
		\فقره  \مهم{کامل بودن}\زیرنویس{\لر{Completeness}}:  در کلاس راننده، همه‌ی اطلاعات مورد نیاز آن ذخیره می‌شود. یک راننده مشخصه‌های مختلفی نظیر نام، نام‌خانوادگی، شماره تماس، تصویر، طول جغرافیایی، عرض جغرافیایی، وضعیت، تاریخ تولد و ... دارد. تمامی این نیازها مطابق با موردهای کاربردی است که در  مرحله‌ی پیاده‌سازی و دیگر مرحله‌ها (مرحله‌ی طراحی و تحلیل) انجام گرفته است. هم‌چنین راننده می‌تواند هر لحظه که می‌خواهد از سامانه رفتارهای مورد نیاز خود را انتظار داشته باشد؛ رفتارهای راننده گزارش حادثه، چاپ کردن اطلاعات بارنامه، گرفتن اطلاعات بار و... است.
		\فقره \مهم{کافی بودن}\زیرنویس{\لر{Sufficiency}}: تمامی ویژگی‌ها و رفتارهای مورد نیاز راننده مطابق با خواسته‌های کارفرما، پیاده‌سازی شده‌اند.
		\فقره \مهم{\لر{Primitiveness}}: کلاس‌ پیاده‌سازی شده برای راننده نیازی به دیگر کلاس‌ها ندارد و عمل‌کرد کاملا مستقلی را دارد. این کار باعث افزایش انسجام\زیرنویس{\لر{cohesion}} و کاهش درهم‌تنیدگی \زیرنویس{\لر{coupling}} می‌شود.
	\پایان{فقرات}
	
	\newpage
	
	\فقره \مهم{کلاس \لر{Customer}}:
	\شروع{فقرات}
	\فقره  \مهم{کامل بودن}: در کلاس صاحب‌ بار، همه‌ی اطلاعات مورد نیاز آن ذخیره می‌شود. یک صاحب بار مشخصه‌های مختلفی نظیر نام، نام‌خانوادگی، شماره تماس، نام کاربری، رمز عبور، تاریخ تولد و ... دارد. تمامی این نیازها مطابق با موردهای کاربردی است که در مرحله‌ی پیاده‌سازی و دیگر مرحله‌ها (مرحله‌ی طراحی و تحلیل) انجام گرفته است. هم‌چنین صاحب بار می‌تواند هر لحظه که می‌خواهد از سامانه رفتارهای مورد نیاز خود را انتظار داشته باشد؛ رفتارهای صاحب بار گرفتن موقعیت جغرافیایی بار، امتیاز دادن به راننده، تایید دریافت بار و... است.
	\فقره \مهم{کافی بودن}: تمامی ویژگی‌ها و رفتارهای مورد نیاز صاحب بار مطابق با خواسته‌های کارفرما، پیاده‌سازی شده‌اند.
	\فقره \مهم{\لر{Primitiveness}}: کلاس‌ پیاده‌سازی شده برای صاحب بار نیازی به دیگر کلاس‌ها ندارد و عمل‌کرد کاملا مستقلی را دارد. این کار باعث افزایش انسجام و کاهش درهم‌تنیدگی می‌شود.
	\پایان{فقرات}
	
	\newpage
	\فقره \مهم{کلاس \لر{Authorizer}}:
	\شروع{فقرات}
		\فقره  \مهم{کامل بودن}: در کلاس مدیر احراز هویت، همه‌ی اطلاعات مورد نیاز آن ذخیره می‌شود. یک مدیر احراز هویت مشخصه‌های مختلفی نظیر نام، نام خانوادگی، شماره تماس، نام کاربری و ... دارد. تمامی این نیاز‌ها مطابق با موردهای کاربردی است که در مرحله‌ی پیاده‌سازی و دیگر مرحله‌ها (مرحله‌ی طراحی و تحلیل) انجام گرفته است، هم‌چنین مدیر احراز هویت می‌توانت هر لحظه که می‌خواهد از سامانه رفتارهای مورد نیاز خود را انتظار داشته باشد؛ رفتارهای دیراحراز هویت تایید راننده، حذف اطلاعات راننده، گرفتن اطلاعات همه‌ی راننده‌ها و ... است.
		\فقره \مهم{کافی بودن}: تمامی ویژگی‌ها و رفتارهای مورد نیاز مدیر احراز هویت مطابق با خواسته‌های کارفرما، پیاده‌سازی شده‌اند.
		\فقره \مهم{\لر{Primitiveness}}: کلاس‌ پیاده‌سازی شده برای مدیر احراز هویت نیازی به دیگر کلاس‌ها ندارد و عمل‌کرد کاملا مستقلی را دارد. این کار باعث افزایش انسجام و کاهش درهم‌تنیدگی می‌شود.
	\پایان{فقرات}
		
	\newpage
	\فقره \مهم{کلاس \لر{SystemAdmin}}
	
	\شروع{فقرات}
		\فقره  \مهم{کامل بودن}: در کلاس مدیر سامانه، همه‌ی اطلاعات مورد نیاز آن ذخیره می‌شود. یک مدیرسامانه مشخصه‌های مختلفی نظیر نام، نام‌خانوادگی، شماره تماس، و ... را دارد. تمامی این نیازها مطابق با موردهای کاربردی است که در  مرحله‌ی پیاده‌سازی و دیگر مرحله‌ها (مرحله‌ی طراحی و تحلیل) انجام گرفته است. هم‌چنین مدیرسامانه می‌تواند هر لحظه که می‌خواهد از سامانه رفتارهای مورد نیاز خود را انتظار داشته باشد؛ اختصاص بار به راننده، اختصاص وسیله نقلیه به بار، اضافه کردن مدیر احراز هویت جدید و... است.
		
		\فقره \مهم{کافی بودن}: تمامی ویژگی‌ها و رفتارهای مورد نیاز  مدیرسامانه مطابق با خواسته‌های کارفرما، پیاده‌سازی شده‌اند.
		\فقره \مهم{\لر{Primitiveness}}: کلاس‌ پیاده‌سازی شده برای مدیر سامانه نیازی به دیگر کلاس‌ها ندارد و عمل‌کرد کاملا مستقلی را دارد. این کار باعث افزایش انسجام و کاهش درهم‌تنیدگی می‌شود.
	\پایان{فقرات}
	\newpage
	\فقره \مهم{کلاس \لر{Order}}

	\شروع{فقرات}
		\فقره  \مهم{کامل بودن}: در کلاس بار، همه‌ی اطلاعات مورد نیاز آن ذخیره می‌شود. یک بار مشخصه‌های مختلفی نظیر وضعیت بار، صاحب بار، وسیله نقلیه، آدرس مقصد  و ... را دارد. تمامی این نیازها مطابق با موردهای کاربردی است که در  مرحله‌ی پیاده‌سازی و دیگر مرحله‌ها (مرحله‌ی طراحی و تحلیل) انجام گرفته است. هم‌چنین بار در هر لحظه رفتارهای متفاوتی در سامانه دارد. حذف کردن سفارش، به‌روزرسانی اطلاعات سفارش  و... است.
		\فقره \مهم{کافی بودن}: تمامی ویژگی‌ها و رفتارهای مورد نیاز  سفارش مطابق با خواسته‌های کارفرما، پیاده‌سازی شده‌اند.
		\فقره \مهم{\لر{Primitiveness}}: کلاس‌ پیاده‌سازی شده برای سفارش نیازی به دیگر کلاس‌ها ندارد و عمل‌کرد کاملا مستقلی را دارد. این کار باعث افزایش انسجام و کاهش درهم‌تنیدگی می‌شود.
	\پایان{فقرات}
	\newpage
	\فقره \مهم{کلاس \لر{Crash}}
	
	\شروع{فقرات}
		\فقره  \مهم{کامل بودن}: در کلاس تصادف، همه‌ی اطلاعات مورد نیاز آن ذخیره می‌شود. یک تصادف مشخصه‌های مختلفی نظیر راننده، توضیحات سانجه و ... را دارد. تمامی این نیازها مطابق با موردهای کاربردی است که در  مرحله‌ی پیاده‌سازی و دیگر مرحله‌ها (مرحله‌ی طراحی و تحلیل) انجام گرفته است. هم‌چنین تصادف در هر لحظه رفتارهای متفاوتی در سامانه دارد. 
		\فقره \مهم{کافی بودن}: تمامی ویژگی‌ها و رفتارهای مورد نیاز تصادف مطابق با خواسته‌های کارفرما، پیاده‌سازی شده‌اند.
		\فقره \مهم{\لر{Primitiveness}}: کلاس‌ پیاده‌سازی شده برای تصادف نیازی به دیگر کلاس‌ها ندارد و عمل‌کرد کاملا مستقلی را دارد. این کار باعث افزایش انسجام و کاهش درهم‌تنیدگی می‌شود.
	\پایان{فقرات}
	\newpage
	\فقره \مهم{کلاس \لر{Vehicle}}
	
	\شروع{فقرات}
		\فقره  \مهم{کامل بودن}: در کلاس وسیله نقلیه، همه‌ی اطلاعات مورد نیاز آن ذخیره می‌شود. یک وسیله نقلیه مشخصه‌های مختلفی نظیر نوع وسیله نقلیه،  شماره پلاک، تعداد وظیفه و ... را دارد. تمامی این نیازها مطابق با موردهای کاربردی است که در  مرحله‌ی پیاده‌سازی و دیگر مرحله‌ها (مرحله‌ی طراحی و تحلیل) انجام گرفته است.
		\فقره \مهم{کافی بودن}: تمامی ویژگی‌ها و رفتارهای مورد نیاز وسیله نقلیه مطابق با خواسته‌های کارفرما، پیاده‌سازی شده‌اند.
		\فقره \مهم{\لر{Primitiveness}}: کلاس‌ پیاده‌سازی شده برای وسیله نقلیه نیازی به دیگر کلاس‌ها ندارد و عمل‌کرد کاملا مستقلی را دارد. این کار باعث افزایش انسجام و کاهش درهم‌تنیدگی می‌شود.
	\پایان{فقرات}
\پایان{فقرات}

