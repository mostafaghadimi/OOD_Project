\قسمت{مستند نصب}

در این قسمت نحوه‌ی نصب و راه‌اندازی قسمت‌های \لر{Back-End} و \لر{Front-End} به طور کامل توضیح داده خواهد شد.

\شروع{فقرات}
	\فقره \مهم{\لر{Back-End}}: برای نصب و راه‌اندازی موفق قسمت \لر{Back-End} پیش‌نهاد می‌شود مراحل زیر را به ترتیب و مطابق با دستورالعمل‌های گفته شده انجام دهید.
	
	\شروع{شمارش}
		\فقره مطمئن شوید نرم‌افزار پایتون را نصب کرده‌اید.
		
		\مهم{نکته:} برای اطمینان از نصب موفقیت‌آمیز و قرار گرفتن پایتون در \لر{environment variables‍} ابتدا ترمینال را باز کرده و سپس دستور \کد{python3 --version} را وارد نمایید.
		
		
		\فقره به کمک دستور زیر \لر{virtual env} را نصب کنید.
		
		\begin{latin}
			\begin{verbatim}
> pip3 install virtualenv
			\end{verbatim}
		\end{latin}
	
	\فقره در مسیر جدید یک \لر{virtualenv} جدید بسازید و سپس آن را فعال کنید. این کار به شما کمک می‌کند تا پکیج‌های مورد نیاز پروژه را در دایرکتوری آن دانلود، نصب و اضافه نمایید.
		\begin{latin}
			\begin{verbatim}
> virtualenv venv
> source venv/bin/activate
			\end{verbatim}
		\end{latin}
	\فقره حال کافی است که پکیج‌های مورد نیاز پروژه که قبلا در فایل \لر{requirements.txt} لیست شده‌اند را نصب کنید.
		\begin{latin}
			\begin{verbatim}
> (venv) pip3 install -r requirements.txt
			\end{verbatim}
		\end{latin}
	\فقره داخل پوشه‌ی \لر{transportation} رفته و برای ساخته شدن جدول‌های پایگاه‌داده و هم‌چنین اجرا شدن سرور، دستورهای زیر را اجرا کنید.
		\begin{latin}
			\begin{verbatim}
> (venv) python3 manage.py makemigrations
> (venv) python3 manage.py migrate
> (venv) python3 manage.py runserver
			\end{verbatim}
		\end{latin}
	
	\پایان{شمارش}

	
\پایان{فقرات}