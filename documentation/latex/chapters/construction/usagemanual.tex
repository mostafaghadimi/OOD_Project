\قسمت{راهنمای استفاده}

با توجه به این که سیستم طراحی شده از بخش‌های مختلفی تشکیل شده است، این راهنما برای کمک به کاربر قرار داده شده تا قسمت‌های مختلف سیستم آشنا شده و به راحتی با آن کار کند.
\شروع{فقرات}
	\فقره \مهم{صاحب بار}
	
	\شروع{شمارش}
		\فقره مطمئن شوید نرم‌افزار پایتون را نصب کرده‌اید.
		
		\مهم{نکته:} برای اطمینان از نصب موفقیت‌آمیز و قرار گرفتن پایتون در \لر{environment variables‍} ابتدا ترمینال را باز کرده و سپس دستور \کد{python3 --version} را وارد نمایید.
		
		\فقره مسیر خود را به پوشه‌ی \لر{back} تغییر دهید.
	
		
		
		
		\فقره به کمک دستور زیر \لر{virtualenv} را نصب کنید.
		

	
	\پایان{شمارش}

	\فقره \مهم{راننده}
	
	\شروع{شمارش}
		\فقره مطمئن شوید \لر{NodeJS} را روی سیستم خود نصب کرده‌اید.
		
		\مهم{نکته:} برای اطمینان از نصب موفقیت‌آمیز و قرار گرفتن \لر{NodeJS} در \لر{environment variables‍} ابتدا ترمینال را باز کرده و سپس دستور \کد{node --version} را وارد نمایید.
	\پایان{شمارش}
	\فقره \مهم{مدیر احراز هویت}
		\شروع{شمارش}
			\فقره برای \مهم{ورود} از منوی جانبی صفحه اصلی گزینه \مهم{مدیر احراز هویت} را انتخاب کنید. سپس می‌توانید با زدن دکمه «ورود» در منوی جانبی، و وارد کردن «نام کاربری» و «رمز عبور» خود، به عنوان یک کاربر \مهم{مدیر احراز هویت} وارد شوید.
			\فقره حال که به عنوان مدیر احراز هویت وارد شدید، می‌توانید دسترسی‌های یک مدیر احراز هویت را مشاهده کنید. این دسترسی‌ها عبارتند از «لیست درخواست‌های ثبت نام»
			\فقره برای اضافه شدن نرم‌افزار باید روی گزینه‌ی \لر{Add to home screen} کلیک کنیم.
		\پایان{شمارش}
	\فقره \مهم{مدیر سامانه}
		\شروع{شمارش}
			\فقره ابتدا باید داخل پوشه 
			\فقره حال کافی است به آدرس \لر{localhost:5000/install} برویم و در آن‌جا روی گزینه‌ی «نصب نرم‌افزار تلفن همراه» کلیک کنیم.
			\فقره برای اضافه شدن نرم‌افزار باید روی گزینه‌ی \لر{Add to home screen} کلیک کنیم.
		\پایان{شمارش}
\پایان{فقرات}