\قسمت{راهنمای استفاده}

با توجه به این که سیستم طراحی شده از بخش‌های مختلفی تشکیل شده است، این راهنما برای کمک به کاربر قرار داده شده تا قسمت‌های مختلف سیستم آشنا شده و به راحتی با آن کار کند. به این منظور یک اجرای کامل از تمامی بخش‌های سامانه با داده‌های فرضی نمایش داده شده است. فهرست خدمات موجود در سامانه به شرح زیر است:
\شروع{شمارش}
    \فقره ورود: به کاربران سامانه اجازه می‌دهد تا در سطح اختیارات خود، از امکانات سامانه استفاده کنند.
    \فقره مشاهده لیست همه سفارش‌های یک کاربر: مدیر سامانه و خود کاربر می‌توانند تاریخچه سفارش‌های کاربر را مشاهده کنند.
    \فقره مشاهده پروفایل یک کاربر: مشاهده اطلاعات عمومی یک کاربر، اعم از نام و نام خانوادگی، عکس و نام کاربری
    \فقره مشاهده لیست رانندگان طرف حساب یک کاربر
    \فقره ثبت‌نام راننده: راننده‌ها می‌توانند در سامانه ثبت‌نام کنند تا در صورت تایید توسط مدیر احراز هویت، از سامانه استفاده کنند.
    \فقره اعلام تصادف یک راننده: راننده‌ها می‌توانند در زیرمنوی کاربری خود اعلام خطا کنند.
    \فقره احراز یا رد رانندگان ثبت نامی: مدیر احراز هویت می‌تواند راننده‌ها را تایید یا رد نماید.
    \فقره مشاهده لیست همه کاربران و تغییر آن‌ها: مدیر کل سامانه می‌تواند تمامی کاربران را مشاهده کند و در صورت لزوم آن‌ها را تغییر دهد، حذف کند یا اضافه کند.
    \فقره افزودن سفارش: مدیر سامانه می‌تواند یک سفارش جدید ایجاد کند.
    \فقره اختصاص سفارش: مدیر سامانه می‌تواند یک راننده را به یک سفارش اختصاص دهد.
    \فقره پایان سفارش: مدیر سامانه می‌تواند ارسال سفارش را تایید کند و به راننده امتیاز دهد..
    \فقره افزودن یک ماشین: مدیر سامانه می‌تواند یک ماشین را به عنوان ماشین استفاده رانندگان شرکت به لیست ماشین‌ها اضافه کند..
    
\پایان{شمارش}
در برخی صفحات، از دکمه‌های «تایید» و «لغو» استفاده شده است که به ترتیب بیانگر تایید اطلاعات و لغو عملیات است. هم‌چنین در زمانی که کاربر وارد سامانه شده است، از طریق زیرمنوی خود می‌تواند با زدن دکمه خروج از سامانه خارج شود. در نهایت، برخی عملیات‌ها پیغام‌های اخطار یا تاییدی را به کاربر نشان می‌دهند که حاوی اطلاعات لازم برای راهنمایی کاربر هستند. در ادامه، به تفکیک نوع کاربر، راهنمای فعالیت آن‌ها آمده است.
\newpage
\شروع{فقرات}
    \فقره \مهم{صاحب بار}
    \شروع{شمارش}
            \فقره صاحب بار، مشتری سامانه است و بارهای خود را در سامانه ثبت می‌کند تا از طریق رانندگان آن‌ها را به مقصد برساند. برای \مهم{ورود} از منوی جانبی صفحه اصلی گزینه \مهم{صاحب بار} را انتخاب کنید. سپس می‌توانید با زدن دکمه «ورود» در منوی جانبی، و وارد کردن «نام کاربری» و «رمز عبور» خود، به عنوان یک کاربر \مهم{صاحب بار} وارد شوید.

    \شروع{شکل}[htbp!]
        \centerimg{./user manual/customer/customer login.png}{.4\linewidth}
        \شرح{دکمه ورود صاحب بار}
        \برچسب{شکل: با زدن ورود، به عنوان صاحب بار وارد شوید}
    \پایان{شکل}

    \شروع{شکل}[htbp!]
        \centerimg{./user manual/customer/customer login pop-up.png}{.5\linewidth}
        \شرح{ورود صاحب بار}
        \برچسب{شکل: اطلاعات خود را وارد کنید و ورود را بزنید}
    \پایان{شکل}
        \فقره حال که به عنوان صاحب بار وارد شدید، می‌توانید دسترسی‌های یک صاحب بار را مشاهده کنید. این دسترسی‌ها عبارتند از «پروفایل»، «مشاهده سفارشات»، و «لیست راننده‌ها» که با رفتن به صفحه هر کدام می‌توانید آن را مشاهده کنید.

    \شروع{شکل}[htbp!]
        \centerimg{./user manual/customer/customer submenu.png}{.4\linewidth}
        \شرح{زیرمنوی دسترسی‌های صاحب بار}
        \برچسب{شکل: زیرمنوی دسترسی‌های صاحب بار}
    \پایان{شکل}
\newpage
        \فقره با رفتن به پروفایل خود می‌توانید اطلاعات کاربری خود را مشاهده کنید.
    \شروع{شکل}[htbp!]
        \centerimg{./user manual/customer/customer profile.png}{.8\linewidth}
        \شرح{اطلاعات کاربری صاحب‌بار}
        \برچسب{شکل: اطلاعات کاربری صاحب بار}
    \پایان{شکل}
    \فقره با رفتن به سفارشات، می‌توانید لیست سفارش‌های خود را مشاهده کنید.
\شروع{شکل}[htbp!]
        \centerimg{./user manual/customer/customer orders list.png}{.8\linewidth}
        \شرح{لیست سفارش‌های صاحب بار}
        \برچسب{شکل: لیست سفارش‌های صاحب بار}
    \پایان{شکل}



    \فقره با رفتن به لیست رانندگان، می‌توانید رانندگانی که پیش از این با صاحب بار تعامل داشته‌اند را مشاهده کنید. هم‌چنین با زدن دکمه‌های مشاهده نقشه، موقعیت مکانی راننده‌های فعال روی نقشه را می‌بینید و با مشاهده تاریخچه، تاریخچه بارهایی که پیش از این برای شما جابه‌جا کرده است را می‌بینید.
\شروع{شکل}[htbp!]
        \centerimg{./user manual/customer/customer drivers list.png}{.8\linewidth}
        \شرح{لیست راننده‌های صاحب بار}
        \برچسب{شکل: لیست راننده‌های صاحب بار}
    \پایان{شکل}
    \پایان{شمارش}
\newpage
    \فقره \مهم{راننده}
    
    \شروع{شمارش}
        \فقره راننده، کارمند سامانه است و بارهای خود را در سامانه ثبت می‌کند تا از طریق رانندگان آن‌ها را به مقصد برساند. در صورتی که پیش از این ثبت نام نکرده‌اید، از منوی جانبی صفحه اصلی گزینه \مهم{راننده} را انتخاب کنید. سپس می‌توانید با زدن دکمه «ثبت نام» در منوی جانبی، وارد صفحه ثبت نام راننده شوید.
    \شروع{شکل}[htbp!]
        \centerimg{./user manual/Driver/Driver login.png}{.3\linewidth}
        \شرح{دکمه ورود و ثبت نام راننده}
        \برچسب{شکل: با زدن ورود، به عنوان راننده وارد شوید و با زدن ثبت نام، وارد صفحه ثبت نام شوید}
    \پایان{شکل}
\شروع{شکل}[htbp!]
        \centerimg{./user manual/Driver/Driver register.png}{.8\linewidth}
        \شرح{صفحه ثبت نام راننده}
        \برچسب{شکل: صفحه ثبت نام راننده}
    \پایان{شکل}
    در صفحه ثبت نام با ورود اطلاعات شخصی خود می‌توانید یک اکانت بسازید. دقت کنید که ایمیل، کد ملی و شماره تلفن باید از فرمت درست تبعیت کنند. پس از ورود درست اطلاعات، می‌توانید دکمه ثبت اطلاعات را بزنید.

    \شروع{شکل}[htbp!]
        \centerimg{./user manual/Driver/Driver register filled form.png}{.5\linewidth}
        \شرح{تکمیل فرم ثبت نام راننده}
        \برچسب{شکل: فرم ثبت نام راننده- فعال}
    \پایان{شکل}
 پس از ورود درست اطلاعات، می‌توانید دکمه ثبت اطلاعات را بزنید و منتظر بمانید تا از طرف مدیر احراز، تایید شوید.
\شروع{شکل}[htbp!]
        \centerimg{./user manual/Driver/Driver register success.png}{.8\linewidth}
        \شرح{ثبت نام موفق راننده}
        \برچسب{شکل: ثبت نام موفق راننده}
    \پایان{شکل}

 
        \فقره برای \مهم{ورود}، از زیرمنوی \مهم{راننده} گزینه «ورود» را انتخاب کنید و با وارد کردن «نام کاربری» و «رمز عبور» خود، به عنوان یک کاربر \مهم{راننده} وارد شوید.

    \شروع{شکل}[htbp!]
        \centerimg{./user manual/Driver/Driver login pop-up.png}{.8\linewidth}
        \شرح{ورود راننده}
        \برچسب{شکل: اطلاعات خود را وارد کنید و ورود را بزنید}
    \پایان{شکل}

    \فقره حال که به عنوان راننده وارد شدید، می‌توانید دسترسی‌های یک راننده را مشاهده کنید. این دسترسی‌ها عبارتند از «پروفایل»، «تاریخچه بار»، و «اعلام حادثه» که با رفتن به صفحه هر کدام می‌توانید آن را مشاهده کنید.

    \شروع{شکل}[htbp!]
        \centerimg{./user manual/driver/driver submenu.png}{.4\linewidth}
        \شرح{زیرمنوی دسترسی‌های راننده}
        \برچسب{شکل: زیرمنوی دسترسی‌های راننده}
    \پایان{شکل}
        \فقره با رفتن به پروفایل خود می‌توانید اطلاعات کاربری خود را مشاهده کنید.
    \شروع{شکل}[htbp!]
        \centerimg{./user manual/driver/driver profile.png}{.8\linewidth}
        \شرح{اطلاعات کاربری راننده}
        \برچسب{شکل: اطلاعات کاربری راننده}
    \پایان{شکل}
    \فقره با رفتن به صفحه تاریخچه بار، می‌توانید لیست سفارش‌های خود و جزئیات مبدا، مقصد و هزینه را مشاهده کنید.
\newpage
                \شروع{شکل}[htbp!]
        \centerimg{./user manual/driver/driver orders list.png}{.8\linewidth}
        \شرح{لیست کل سفارش‌های راننده}
        \برچسب{شکل: لیست کل سفارش‌های راننده}
    \پایان{شکل}

    \فقره در صورتی که حین حمل بار یا با ماشین شرکت دچار حادثه شدید، از طریق صفحه اعلام حادثه این موضوع را اطلاع دهید. در بخش توضیحات جزئیات حادثه را بیان کنید.
                \شروع{شکل}[htbp!]
        \centerimg{./user manual/driver/driver accident.png}{.8\linewidth}
        \شرح{اعلام حادثه راننده}
        \برچسب{شکل: اعلام حادثه راننده}

\پایان{شکل}
    \پایان{شمارش}
    \فقره \مهم{مدیر احراز هویت}
        \شروع{شمارش}
            \فقره مدیر احراز هویت، مسئول تایید رانندگانی است  برای \مهم{ورود} از منوی جانبی صفحه اصلی گزینه \مهم{مدیر احراز هویت} را انتخاب کنید. سپس می‌توانید با زدن دکمه «ورود» در منوی جانبی، و وارد کردن «نام کاربری» و «رمز عبور» خود، به عنوان یک کاربر \مهم{مدیر احراز هویت} وارد شوید.

    \شروع{شکل}[htbp!]
        \centerimg{./user manual/authorizer/authorizer login.png}{.3\linewidth}
        \شرح{دکمه ورود مدیر احراز هویت}
        \برچسب{شکل: با زدن ورود، به عنوان مدیر احراز هویت وارد شوید}
    \پایان{شکل}

    \شروع{شکل}[htbp!]
        \centerimg{./user manual/authorizer/authorizer login pop-up.png}{.8\linewidth}
        \شرح{ورود مدیر احراز هویت}
        \برچسب{شکل: اطلاعات خود را وارد کنید و ورود را بزنید}
    \پایان{شکل}
\newpage
            \فقره حال که به عنوان مدیر احراز هویت وارد شدید، می‌توانید دسترسی‌های یک مدیر احراز هویت را مشاهده کنید. این دسترسی‌ها عبارتند از «لیست درخواست‌های ثبت نام» و با رفتن به صفحه آن، می‌توانید لیست درخواست‌های احراز هویت رانندگان را مشاهده کنید.

    \شروع{شکل}[htbp!]
        \centerimg{./user manual/authorizer/authorizer submenu.png}{.5\linewidth}
        \شرح{لیست دسترسی‌های مدیر احراز هویت}
        \برچسب{شکل: دسترسی‌های مدیر احراز هویت}
    \پایان{شکل}

\شروع{شکل}[htbp!]
        \centerimg{./user manual/authorizer/authorizer driver menu.png}{.8\linewidth}
        \شرح{لیست رانندگان برای تایید یا رد}
        \برچسب{شکل: لیست رانندگان ثبت نام کرده}
    \پایان{شکل}
\newpage
            \فقره حال با زدن دکمه‌های رد یا تایید مربوط به هر راننده‌، می‌توانید آن راننده را احراز و یا رد کنید.
\شروع{شکل}[htbp!]
        \centerimg{./user manual/authorizer/authorizer authorize buttons.png}{.3\linewidth}
        \شرح{دکمه رد و تایید راننده}
        \برچسب{شکل: دکمه رد و تایید راننده}
    \پایان{شکل}

\شروع{شکل}[htbp!]
        \centerimg{./user manual/authorizer/authorizer authorize success.png}{.5\linewidth}
        \شرح{تایید راننده}
        \برچسب{شکل: تایید موفقیت‌آمیز راننده}
    \پایان{شکل}

\شروع{شکل}[htbp!]
        \centerimg{./user manual/authorizer/authorizer authorize fail.png}{.5\linewidth}
        \شرح{رد راننده}
        \برچسب{شکل: رد راننده}
    \پایان{شکل}
        \پایان{شمارش}
\newpage
    \فقره \مهم{مدیر سامانه}
        \شروع{شمارش}
            \فقره مدیر سامانه، بررسی جزئیات کل سامانه و در صورت نیاز اصلاح داده‌های دیگران را بر عهده دارد. برای \مهم{ورود} از منوی جانبی صفحه اصلی گزینه \مهم{مدیر سامانه} را انتخاب کنید. سپس می‌توانید با زدن دکمه «ورود» در منوی جانبی، و وارد کردن «نام کاربری» و «رمز عبور» خود، به عنوان یک کاربر \مهم{مدیر سامانه} وارد شوید.

    \شروع{شکل}[htbp!]
        \centerimg{./user manual/admin/admin login.png}{.3\linewidth}
        \شرح{دکمه ورود مدیر سامانه}
        \برچسب{شکل: با زدن ورود، به عنوان مدیر سامانه وارد شوید}
    \پایان{شکل}

    \شروع{شکل}[htbp!]
        \centerimg{./user manual/admin/admin login pop-up.png}{.8\linewidth}
        \شرح{ورود مدیر سامانه}
        \برچسب{شکل: اطلاعات خود را وارد کنید و ورود را بزنید}
    \پایان{شکل}

            \فقره حال که به عنوان مدیر سامانه وارد شدید، می‌توانید دسترسی‌های یک مدیر سامانه را مشاهده کنید. این دسترسی‌ها عبارتند از مشاهده و تغییر «لیست مدیران احراز هویت»، «لیست مشتریان»، «لیست سفارش‌ها» و «لیست رانندگان» و با رفتن به صفحه آن‌ها، می‌توانید این لیست‌ها را مشاهده کنید و در صورت نیاز تغییر دهید.

    \شروع{شکل}[htbp!]
        \centerimg{./user manual/admin/admin submenu.png}{.5\linewidth}
        \شرح{لیست دسترسی‌های مدیر سامانه}
        \برچسب{شکل: دسترسی‌های مدیر سامانه}
    \پایان{شکل}
    \فقره با رفتن به صفحه «لیست مشتریان» می‌توانید لیست صاحبان بار را مشاهده کنید، و در صورت نیاز یک مشتری را حذف، اضافه یا ویرایش کنید.

    \شروع{شکل}[htbp!]
        \centerimg{./user manual/admin/admin customers list.png}{.5\linewidth}
        \شرح{مشاهده لیست مشتریان توسط مدیر سامانه}
        \برچسب{شکل: مشاهده لیست مشتریان توسط مدیر سامانه}
        \پایان{شکل}
\newpage
        \فقره با زدن دکمه «اضافه کردن» صفحه‌ای برای شما باز می‌شود که در آن می‌توانید یک مشتری جدید اضافه کنید. تا زمانی که اطلاعات کامل را وارد نکنید اجازه اضافه کردن ندارید.
\شروع{شکل}[htbp!]
        \centerimg{./user manual/admin/admin add customer.png}{.5\linewidth}
        \شرح{فرم افزودن مشتریان توسط مدیر سامانه}
        \برچسب{شکل: فرم افزودن مشتریان توسط مدیر سامانه}
    \پایان{شکل}
\\
پس از ورود اطلاعات، می‌توانید با زدن دکمه اضافه کردن این مشتری را به سامانه اضافه کنید.
\شروع{شکل}[htbp!]
        \centerimg{./user manual/admin/admin add customer filled.png}{.5\linewidth}
        \شرح{ افزودن مشتریان توسط مدیر سامانه}
        \برچسب{شکل: افزودن مشتریان توسط مدیر سامانه}
    \پایان{شکل}
\\
در صورت موفقیت‌آمیز بودن این اقدام، پیامی دریافت خواهید کرد که نشان می‌دهد این عمل انجام گرفته است.
\فقره با زدن دکمه ویرایش در هر سطر (که با یک مداد نمایش داده شده است) صفحه‌ای برای شما باز می‌شود که در آن می‌توانید این مشتری را ویرایش کنید. سطرهای مورد نیاز را ویرایش کنید و دکمه «ویرایش» را بزنید.
\شروع{شکل}[htbp!]
        \centerimg{./user manual/admin/admin edit customer.png}{.5\linewidth}
        \شرح{ویرایش مشتری}
        \برچسب{شکل: ویرایش مشتری}
\پایان{شکل}
    \فقره با زدن دکمه پاک کردن (که با یک علامت منفی نمایش داده شده است) می‌توانید یک مشتری را حذف کنید. در این صورت با پیام زیر، موفقیت شما نمایش داده می‌شود.
\شروع{شکل}[htbp!]
        \centerimg{./user manual/admin/admin delete customer success.png}{.5\linewidth}
        \شرح{حذف مشتری}
        \برچسب{شکل: حذف مشتری}
\پایان{شکل}
            \فقره با رفتن به صفحه «لیست مدیران احراز هویت» می‌توانید لیست مدیران را مشاهده کنید، و در صورت نیاز مدیری را حذف، اضافه یا ویرایش کنید.

    \شروع{شکل}[htbp!]
        \centerimg{./user manual/admin/admin authorizer list.png}{.5\linewidth}
        \شرح{مشاهده لیست مدیران احراز هویت توسط مدیر سامانه}
        \برچسب{شکل: مشاهده لیست مدیران احراز هویت توسط مدیر سامانه}
        \پایان{شکل}
\newpage
        \فقره با زدن دکمه «اضافه کردن» صفحه‌ای برای شما باز می‌شود که در آن می‌توانید یک مدیر احراز هویت جدید اضافه کنید. تا زمانی که اطلاعات کامل را وارد نکنید اجازه اضافه کردن ندارید.
\شروع{شکل}[htbp!]
        \centerimg{./user manual/admin/admin add authorizer - 1.png}{.5\linewidth}
        \شرح{فرم افزودن مدیران احراز هویت توسط مدیر سامانه}
        \برچسب{شکل: فرم افزودن مدیران احراز هویت توسط مدیر سامانه}
    \پایان{شکل}
\\
پس از ورود اطلاعات، می‌توانید با زدن دکمه اضافه کردن این مدیر را به سامانه اضافه کنید.
\شروع{شکل}[htbp!]
        \centerimg{./user manual/admin/admin add authorizer - 2.png}{.5\linewidth}
        \شرح{ افزودن مدیران احراز هویت توسط مدیر سامانه}
        \برچسب{شکل: افزودن مدیران احراز هویت توسط مدیر سامانه}
    \پایان{شکل}
\\
در صورت موفقیت‌آمیز بودن این اقدام، پیامی دریافت خواهید کرد که نشان می‌دهد این عمل انجام گرفته است.
\فقره با زدن دکمه ویرایش در هر سطر (که با یک مداد نمایش داده شده است) صفحه‌ای برای شما باز می‌شود که در آن می‌توانید این مدیر احراز هویت را ویرایش کنید. سطرهای مورد نیاز را ویرایش کنید و دکمه «ویرایش» را بزنید.
\شروع{شکل}[htbp!]
        \centerimg{./user manual/admin/admin edit authorizer.png}{.5\linewidth}
        \شرح{ویرایش مدیر احراز هویت}
        \برچسب{شکل: ویرایش مدیر احراز هویت}
\پایان{شکل}

    \فقره مانند لیست مشتریان، می‌توانید با زدن دکمه پاک کردن (که با یک علامت منفی نمایش داده شده است) یک مدیر احراز را حذف کنید و برایش پیام مناسب نمایش داده می‌شود.
    \فقره با رفتن به صفحه «لیست رانندگان» می‌توانید لیست رانندگان را مشاهده کنید، و در صورت نیاز راننده‌ای را حذف، اضافه یا ویرایش کنید.


    \شروع{شکل}[htbp!]
        \centerimg{./user manual/admin/admin Drivers list.png}{.5\linewidth}
        \شرح{مشاهده لیست رانندگان توسط مدیر سامانه}
        \برچسب{شکل: مشاهده لیست رانندگان توسط مدیر سامانه}
        \پایان{شکل}
\newpage
        \فقره با زدن دکمه «اضافه کردن» صفحه‌ای برای شما باز می‌شود که در آن می‌توانید یک راننده جدید اضافه کنید. تا زمانی که اطلاعات کامل را وارد نکنید اجازه اضافه کردن ندارید.
\شروع{شکل}[htbp!]
        \centerimg{./user manual/admin/admin add driver.png}{.5\linewidth}
        \شرح{فرم افزودن راننده توسط مدیر سامانه}
        \برچسب{شکل: فرم افزودن راننده توسط مدیر سامانه}
    \پایان{شکل}
\\
پس از ورود اطلاعات، می‌توانید با زدن دکمه اضافه کردن این مدیر را به سامانه اضافه کنید.
\شروع{شکل}[htbp!]
        \centerimg{./user manual/admin/admin add driver filled form.png}{.5\linewidth}
        \شرح{ افزودن راننده توسط مدیر سامانه}
        \برچسب{شکل: افزودن راننده توسط مدیر سامانه}
    \پایان{شکل}
\\
در صورت موفقیت‌آمیز بودن این اقدام، پیامی دریافت خواهید کرد که نشان می‌دهد این عمل انجام گرفته است.
\فقره با زدن دکمه ویرایش در هر سطر (که با یک مداد نمایش داده شده است) صفحه‌ای برای شما باز می‌شود که در آن می‌توانید این راننده را ویرایش کنید. سطرهای مورد نیاز را ویرایش کنید و دکمه «ویرایش» را بزنید. این قسمت مشابه لیست مشتریان و مدیران احراز هویت است.
    \فقره مانند لیست مشتریان و مدیران احراز هویت، می‌توانید با زدن دکمه پاک کردن (که با یک علامت منفی نمایش داده شده است) یک مدیر احراز را حذف کنید و برایش پیام مناسب نمایش داده می‌شود.
    \فقره با مشاهده هر کدام از دکمه‌های «مشاهده موقعیت جغرافیایی» و «مشاهده تاریخچه» صفحه مورد نظر از راننده باز می‌شود و می‌توانید اطلاعات آن را مشاهده کنید.
\شروع{شکل}[htbp!]
        \centerimg{./user manual/admin/admin view driver history.png}{.5\linewidth}
        \شرح{مشاهده تاریخچه راننده توسط مدیر سامانه}
        \برچسب{شکل: مشاهده تاریخچه راننده توسط مدیر سامانه}
    \پایان{شکل}

    \فقره با رفتن به صفحه «لیست سفارشات» می‌توانید لیست سفارشات را مشاهده کنید، و در صورت نیاز سفارشی را حذف، اضافه یا ویرایش کنید.
    \شروع{شکل}[htbp!]
        \centerimg{./user manual/admin/admin orders list.png}{.5\linewidth}
        \شرح{مشاهده لیست سفارش‌ها توسط مدیر سامانه}
        \برچسب{شکل: مشاهده لیست سفارش ها توسط مدیر سامانه}
        \پایان{شکل}
\newpage
        \فقره با زدن دکمه «اضافه کردن» صفحه‌ای برای شما باز می‌شود که در آن می‌توانید یک سفارش جدید اضافه کنید.
\شروع{شکل}[htbp!]
        \centerimg{./user manual/admin/admin add order.png}{.5\linewidth}
        \شرح{فرم افزودن سفارش توسط مدیر سامانه}
        \برچسب{شکل: فرم افزودن سفارش توسط مدیر سامانه}
    \پایان{شکل}
\\
در صورت موفقیت‌آمیز بودن این اقدام، پیامی دریافت خواهید کرد که نشان می‌دهد این عمل انجام گرفته است.
    \فقره مانند لیست مشتریان و مدیران احراز هویت، می‌توانید با زدن دکمه پاک کردن (که با یک علامت منفی نمایش داده شده است) یک مدیر احراز را حذف کنید و برایش پیام مناسب نمایش داده می‌شود.
    \فقره با فشردن هر کدام از دکمه‌های ستون‌های «راننده» و «ماشین» می‌توانید به یک سفارش که راننده یا ماشین ندارد، یک راننده و یک ماشین اختصاص دهید.
\شروع{شکل}[htbp!]
        \centerimg{./user manual/admin/admin assign driver.png}{.5\linewidth}
        \شرح{اختصاص یک راننده به سفارش توسط مدیر سامانه}
        \برچسب{شکل: اختصاص یک راننده به سفارش توسط مدیر سامانه}
    \پایان{شکل}

\شروع{شکل}[htbp!]
        \centerimg{./user manual/admin/admin assign vehicle.png}{.5\linewidth}
        \شرح{اختصاص یک ماشین به سفارش توسط مدیر سامانه}
        \برچسب{شکل: اختصاص یک ماشین به سفارش توسط مدیر سامانه}
    \پایان{شکل}
\فقره پس از پایان سفر، می‌توانید در ستون «تایید ارسال» تحویل سفارش را تایید کنید و به عملکرد راننده امتیاز دهید.

\شروع{شکل}[htbp!]
        \centerimg{./user manual/admin/admin score order.png}{.5\linewidth}
        \شرح{تایید سفارش و امتیازدهی توسط مدیر سامانه}
        \برچسب{شکل: تایید سفارش و امتیازدهی توسط مدیر سامانه}
    \پایان{شکل}
\فقره پس از تخصیص راننده، تخصیص ماشین یا اتمام سفر نمی‌توانید دوباره همان عمل را انجام دهید؛ این اعمال تنها یک بار برای هر سفارش اعمال می‌شوند و سپس دکمهٔ آن‌ها غیر فعال می‌شود.

\شروع{شکل}[htbp!]
        \centerimg{./user manual/admin/admin orders list 2.png}{.5\linewidth}
        \شرح{تایید سفارش و امتیازدهی توسط مدیر سامانه}
        \برچسب{شکل: تایید سفارش و امتیازدهی توسط مدیر سامانه}
    \پایان{شکل}


    \فقره با رفتن به صفحه «لیست ماشین‌ها» می‌توانید لیست ماشین‌ها را مشاهده کنید، و در صورت نیاز ماشینی را حذف، اضافه یا ویرایش کنید.
    \شروع{شکل}[htbp!]
        \centerimg{./user manual/admin/admin vehicles list.png}{.5\linewidth}
        \شرح{مشاهده لیست ماشین‌ها توسط مدیر سامانه}
        \برچسب{شکل: مشاهده لیست ماشین‌ها توسط مدیر سامانه}
        \پایان{شکل}
\newpage
        \فقره با زدن دکمه «اضافه کردن» صفحه‌ای برای شما باز می‌شود که در آن می‌توانید یک ماشین جدید اضافه کنید. تا زمانی که اطلاعات کامل را وارد نکنید اجازه اضافه کردن ندارید.
\شروع{شکل}[htbp!]
        \centerimg{./user manual/admin/admin add vehicle.png}{.5\linewidth}
        \شرح{فرم افزودن ماشین توسط مدیر سامانه}
        \برچسب{شکل: فرم افزودن ماشین توسط مدیر سامانه}
    \پایان{شکل}
\\
پس از ورود اطلاعات، می‌توانید با زدن دکمه اضافه کردن این مدیر را به سامانه اضافه کنید.
\شروع{شکل}[htbp!]
        \centerimg{./user manual/admin/admin add vehicle filled.png}{.5\linewidth}
        \شرح{ افزودن ماشین توسط مدیر سامانه}
        \برچسب{شکل: افزودن ماشین توسط مدیر سامانه}
    \پایان{شکل}
\\
در صورت موفقیت‌آمیز بودن این اقدام، پیامی دریافت خواهید کرد که نشان می‌دهد این عمل انجام گرفته است.
\فقره با زدن دکمه ویرایش در هر سطر (که با یک مداد نمایش داده شده است) صفحه‌ای برای شما باز می‌شود که در آن می‌توانید این ماشین را ویرایش کنید. سطرهای مورد نیاز را ویرایش کنید و دکمه «ویرایش» را بزنید. این قسمت مشابه لیست مشتریان و مدیران احراز هویت است.
    \فقره مانند لیست مشتریان و مدیران احراز هویت، می‌توانید با زدن دکمه پاک کردن (که با یک علامت منفی نمایش داده شده است) یک مدیر احراز را حذف کنید و برایش پیام مناسب نمایش داده می‌شود.
    .
    \فقره با فشردن هر کدام از دکمه‌های ستون‌های «مشاهده تاریخچه» و «مشاهده تصادفات ماشین» می‌توانید تاریخچه یا تصادفات ماشین نظیر آن سطر را مشاهده کنید.
\شروع{شکل}[htbp!]
        \centerimg{./user manual/admin/admin show vehicle accidents.png}{.5\linewidth}
        \شرح{نمایش حوادث یک ماشین توسط مدیر سامانه}
        \برچسب{شکل: نمایش تصادفات یک ماشین توسط مدیر سامانه}
    \پایان{شکل}

\شروع{شکل}[htbp!]
        \centerimg{./user manual/admin/admin vehicle history.png}{.5\linewidth}
        \شرح{نمایش تاریخچه یک ماشین توسط مدیر سامانه}
        \برچسب{شکل: نمایش تاریخچه یک ماشین توسط مدیر سامانه}
    \پایان{شکل}

        \پایان{شمارش}
\clearpage
\پایان{فقرات}

